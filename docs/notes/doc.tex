\documentclass[10pt,a4paper]{report}
 \usepackage{booktabs}
\usepackage{float}
\usepackage{tabularx}
\usepackage{amsmath}
\usepackage{amsfonts}
\usepackage{amssymb}
\usepackage{graphicx}


\usepackage{polski}
\usepackage[polish]{babel}
\usepackage[utf8]{inputenc}
\usepackage[polish]{babel}

\usepackage{hyperref}
\hypersetup{
	colorlinks=true, %set true if you want colored links
	linktoc=all,     %set to all if you want both sections and subsections linked
	linkcolor=red,  %choose some color if you want links to stand out
}

\usepackage{tikz}
\def\checkmark{\tikz\fill[scale=0.4](0,.35) -- (.25,0) -- (1,.7) -- (.25,.15) -- cycle;} 
\usepackage{color, colortbl}



\usepackage{geometry}
\geometry{legalpaper, margin=1in}

\newcommand{\img}[4]{
	\begin{figure}[H]
		\begin{center}
			\includegraphics[width=#1 cm, keepaspectratio=true]{#2}
			\caption{#3}
			\label{#4}
		\end{center}
	\end{figure}
}

\definecolor{Gray}{gray}{0.9}
\setcounter{secnumdepth}{4}

\title{Notatki do pracy inżynierskiej. \\ 8-bitowy przetwornik cyfrowo-analogowy. \\ STAN : pre-alpha}
\date{2018}
\author{Michał Czyż \\ student, WETI PG}

\begin{document}
	\maketitle
	
	\tableofcontents
	\newpage
	\chapter{Preliminaria.}
	\section{Abstract.}
	\section{Abstrakt.}
	W pracy został zaprojektowany taki-a-taki układ, który jest super!
	\section{Podziękowania.}
	
	\chapter{Przetworniki cyfrowo-analogowe.}
	\section{Wstęp.}
	Tematem pracy jest przetwornik C/A, który:
	\begin{itemize}
		\item zasilanie 1.8V
		\item CMOS 180nm
		\item czas konwersji poniżej 4ms
		\item 8bitowy
	\end{itemize}

	\subsection{Zasada działania.}
	Zadaniem przetwornika cyfrowo-analogowego jest konwersja sygnału elektrycznego z dziedziny cyfrowej do analogowej. W tej pracy układ przetwarza 8-bitowe liczby na wartość napięcia od zera do zasilania.
	
	Sygnał wejściowy to w ogólności N-bitowe słowo $B_{in}={b_{n-1},b_{n-2},...,b_0}$. Wszystkie możliwe N-bitowe liczby tworzą przestrzeń $D$. Napięcie wyjściowe może przyjąć każdą wartość z zakresu napięć zasilania $V_{out}=f(B_{in})$, gdzie:

	\begin{align}
	\mathcal{D} \ni  \mathcal{B}_{in} &\to \mathcal{V}_{out} \in \left(V_{gnd}, V_{ref} \right)
	\end{align}
	
	
	\img{20}{../visio/blokschkonca.pdf}{Blokowy schemat konwertera C/A.}{blokschkonca}
	
	Nie jest to pełny obrazek, może być potrzebny blok S\&H na wyjściu, jeżeli konwersja mocno szumi i przechodzi przez wiele stanów.
	
	\subsection{Podział konwerterów.}
	Za \cite{cmosanal} i \cite{integconv} konwertery C/A dzieli sie na:
	\begin{itemize}
		\item statyczne i dynamiczne
		\item rownolegle i szeregowe
		\item nadprobkujace ( w tym delta-sigma )
		\item rampfunction, dual-slope?
	\end{itemize}
	Statyczny - po wpisaniu liczby do rejestru wejściowego napięcie wyjściowe jest utrzymywane do zaniku zasilania.
	Dynamiczny - przeciwnie niż statyczny, napięcie wyjściowe po pewnym czasie zanika, np. ze wzgledu na rozładowanie kondensatorów, więc wymagają odświeżania.
	Rownoległy przetwarza każdy bit rejestru wejściowego jednocześnie.
	Szeregowy przetwarza zawartość rejestru wejściowego bit po bicie.
	Nadprobkujacy, rampfunction, dual-slope???
	
	
	\section{Przestrzeń projektowa.}
	Znając już mnogość klasyfikacji należy się zastanowić co powoduje, że dany konwerter będzie się lepiej nadawał w danej aplikacji, więc trzeba wprowadzić miary jakości i sposoby testowania/mierzenia.
	\subsection{Miary jakości i testy.}
	Konwerter moze zostac scharakteryzowany przez
	\begin{itemize}
		\item Zakres napięć wyjściowych/zasilania
		\item STATIC 
		\subitem Rozdzielczosc
		\subitem efektywna liczba bitów 9.1-13 z \cite{cmosanal} ???
		\item STATIC ERROR
		\subitem offset error
		\subitem gain error
		\subitem quantization error
		\subitem integral nonlinearity
		\subitem differential nonlinearity
		\subitem monotonicity
		\item DYNAMIC
		\subitem conversion speed
		\subitem 
	\end{itemize}
	Na poziomie symulacji można je wyznaczyć, więc trzeba przygotować stanowiska pomiarowe.
	
	Tzw. test in-out wymaga stworzenia modelu generatora słow cyfrowych, dac jako dut, modelu adc do przetwarzania wyjscia dac i pomiaru roznicy liczby wejsciowej z wyjsciowa adc. Problem: jak stworzyc model adc? Pomiar uznaje sie za dobry, gdy adc jest n+2 bitowe. Model adc jest niepotrzebny, bo symulacja daje wynik jako 32-bitowe float.
	
	Test widmowy - generator funkcji cyfrowy. wyjście na analizator widmowy - analiza transient daca i potem fourier
	
	
	
	  
	\chapter{Schemat elektryczny.}
	\section{Schemat/opis.}
	\subsection{Dobór kryteriów prowadzących do tego rozwiązania.}
	\subsection{Schemat.}
	\section{Dowód poprawności działania.}

	\chapter{Pomiary parametrów.}
	\section{środowisko.}
	\section{Układ testujący do wyznaczenia x.}
	\section{Układ testujący do wyznaczenia y.}
	\section{Układ testujący do wyznaczenia z.}
	
	\chapter{Realizacja topografii.}
	\section{środowisko.}
	\section{Technologia.}
	\section{Hierarchizacja projektu.}
	\subsection{Moduł 1.}
	\subsection{Moduł 2.}
	\subsection{Moduł 3.}
	\subsection{Moduł 4.}
	
	\section{Pomiary parametrów po ekstrakcji.}
	\subsection{Sposób pomiaru}
	Jak przed ekstrakcją.
	\subsection{Rozmiar układu, liczba tranzystorów.}

	\chapter{Podsumowanie}
	\section{Weryfikacja osiągniętych celów.}
	
	
	\appendix
	\chapter{Synteza rejestru cyfrowego przy pomocy Cadence Virtuoso.}
	\chapter{Badanie wzmacniacza (a może czegoś innego) analogowego z Cadence.}
	
.
	\begin{thebibliography}{9}
		%%moje z knovela
		\bibitem{integconv} 
		Jespers, Paul G.A.. (2001). 
		\textit{Integrated Converters - D to A and A to D Architectures, Analysis and Simulation.}
		Oxford University Press
		
		\bibitem{cmosanal} 
		Allen, Phillip E. Holberg, Douglas R.. (2012)  
		\textit{CMOS Analog Circuit Design (3rd Edition). }
		Oxford University Press.
		
		\bibitem{vlsidesign} 
		Das, Debaprasad. (2015). .
		\textit{VLSI Design (2nd edition)}
		Oxford University Press.
		%%z opisu tematu od profa allen cmos sie pokrywa z moim wyborem
		\bibitem{plassche} 
		R. Plassche (2001). 
		\textit{Scalone przetworniki analogowo-cyfrowe i cyfrowo-analogowe,}
		WKŁ 2001
	\end{thebibliography}

\end{document}
