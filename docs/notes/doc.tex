\documentclass[10pt,a4paper]{report}
 \usepackage{booktabs}
\usepackage{float}
\usepackage{tabularx}
\usepackage{amsmath}
\usepackage{amsfonts}
\usepackage{amssymb}
\usepackage{graphicx}


\usepackage{polski}
\usepackage[polish]{babel}
\usepackage[utf8]{inputenc}
\usepackage[polish]{babel}

\usepackage{indentfirst}

\usepackage{hyperref}
\hypersetup{
	colorlinks=true, %set true if you want colored links
	linktoc=all,     %set to all if you want both sections and subsections linked
	linkcolor=red,  %choose some color if you want links to stand out
}

\usepackage{tikz}
\def\checkmark{\tikz\fill[scale=0.4](0,.35) -- (.25,0) -- (1,.7) -- (.25,.15) -- cycle;} 
\usepackage{color, colortbl}

\usepackage{geometry}
\geometry{legalpaper, margin=1in}

\newcommand{\img}[4]{
	\begin{figure}[H]
		\begin{center}
			\includegraphics[width=#1 cm, keepaspectratio=true]{#2}
			\caption{#3}
			\label{#4}
		\end{center}
	\end{figure}
}

\definecolor{Gray}{gray}{0.9}
\setcounter{secnumdepth}{4}

\title{Notatki do pracy inżynierskiej. \\ 8-bitowy przetwornik cyfrowo-analogowy. \\ STAN : alpha}
\date{2018}
\author{Michał Czyż \\ student, WETI PG}

\begin{document}
	\maketitle
	
	\tableofcontents
	\newpage
	\chapter{Preliminaria.}
	\section{Abstract.}
	{	Goal of this thesis... \textcolor{red}{NA KONIEC UZUPEŁNIĆ.} }

	\section{Abstrakt.}
	{	Celem niniejszej pracy jest zaprojektowanie schematu i topografii masek 8-bitowego przetwornika cyfrowo analogowego w technologii CMOS AMS 180nm. \textcolor{red}{NA KONIEC UZUPEŁNIĆ.} }
	
	\section{Podziękowania.}
	{	Dziękujemy wszystkim ... . \textcolor{red}{NA KONIEC UZUPEŁNIĆ} }
	
	\chapter{Wstęp.}
	{	Konwertery danych z postaci cyfrowej na analogową (i odwrotnie) są niezbędną częścią systemów elektronicznych, ponieważ umożliwiają komunikację pomiędzy zewnętrznym, analogowym światem i cyfrowymi rdzeniami układów krzemowych \cite{integconv}. Do przykładowych zastosowań konwerterów należą m.in. generowanie sygnału wideo, audio, sterowanie silnikami, radarami. }
	
	{	Dotychczas powstały architektury konwerterów \cite{integconv} \cite{cmosanal}:
		\begin{itemize}
			\item szeregowa. Konwersja odbywa się bit po bicie. Z reguły wolniejsze, ale pobierają mniej mocy.
			\item równoległa. Konwersja odbywa się dla wszystkich bitów równolegle.
			\item segmentowe - z wykorzystaniem "pod-konwerterów".
			\item Nadpróbkujące, również nazywane konwerterami Nyquista. Wolniejsze, ale dokładne.
			\item Konwertery skalowane \textcolor{red}{(czy skalujące?)} wykorzystują ważone odniesienia w postaci prądów, napięć lub ładunków.
			\item Statyczne. Sygnał wyjściowy jest stały, jeżeli nie zmienia się syngał wejściowy, przeciwnie do: dynamiczne.
		\end{itemize} }

	{	Do podstawowych parametrów przetwornika należą (jest tego dużo więcej):
		\begin{itemize}
			\item Dokładność bezwzględna.
			\item Dokładność względna.
			\item Nieliniowość różniczkowa (DNL).
			\item Nieliniowość całkowa (INL). 
			\item Rozdzielczość.
			\item Przesunięcie zera.
			\item Współczynnik wrażliwości temperaturowej.
			\item Stosunek sygnału do szumu.
			\item Monotoniczność.
			\item Maksymalna częstotliwośc próbkowania,
			\item PSRR.
		\end{itemize} 
	}

	{	Kluczowym dla poprawnego działania konwertera skalowanego jest stabilne źródło napięcia odniesienia. Układy wytwarzane w technologii CMOS są wrażliwe na rorzut technologiczny parametrów elementów elektronicznych, zmieniają się ich właściwości pod wpływem zmian temperatury lub zmian wartości napięcia zasilania. Aby zapewnić stabilność napięcia odniesienia stosuje się układ bandgap. }

	\chapter{Zaprojektowany konwerter.}
	{	Narzucone wymagania na przetwornik:
		\begin{itemize}
			\item zasilanie 1.8V,
			\item CMOS AMS 180nm,
			\item szybkość konwersji powyżej 4MS,
			\item 8-bitowy.
		\end{itemize} }
	
	{	Dodatkowe wymagania:
		\begin{itemize}
			\item Sygnał wyjściowy to napięcie o zakresie co najmniej 0-1V.
			\item Dopuszczalne obciążenie: rezystancyjne co najmniej $50\Omega$.
		\end{itemize}
	}

	{	Do zrealizowania wybrano konwerter skalujący prąd. Ogólny schemat takiego konwertera:
		\img{20}{../visio/currentscale.pdf}{Schemat konwertera skalującego prąd}{currentscale}
	}

	{	Podstawowe i niezbędne bloki konwertera skalującego prąd to:
		\begin{itemize}
			\item Blok konwersji.
			\item Rejestr wejściowy.
			\item Źródło napięcia odniesienia typu Bandgap.
		\end{itemize}
		Dodatkowymi blokami mogą być:
		\begin{itemize}
			\item Cyfrowy interfejs, np. SPI, I2C.
			\item Blok kontroli (wysyłanie komend przez interfejs)
			\item Programowalne wzmocnienie wyjścia.
			\item Blok autokalibracji.
		\end{itemize}	
	}
	{
		
	}
	Zadaniem przetwornika cyfrowo-analogowego jest konwersja sygnału elektrycznego z dziedziny cyfrowej do analogowej.
	
	
	\img{20}{../visio/blokschkonca.pdf}{Blokowy schemat konwertera C/A.}{blokschkonca}
	
	Nie jest to pełny obrazek, może być potrzebny blok S\&H na wyjściu, jeżeli konwersja mocno szumi i przechodzi przez wiele stanów. \textcolor{red}{Podać referencję.}
	

	
	
	  
	\chapter{Schemat elektryczny.}
	\section{Schemat/opis.}
	\subsection{Dobór kryteriów prowadzących do tego rozwiązania.}
	\subsection{Schemat.}
	\section{Dowód poprawności działania.}

	\chapter{Pomiary parametrów.}
	\section{środowisko.}
	\section{Układ testujący do wyznaczenia x.}
	\section{Układ testujący do wyznaczenia y.}
	\section{Układ testujący do wyznaczenia z.}
	
	\chapter{Realizacja topografii.}
	\section{środowisko.}
	\section{Technologia.}
	\section{Hierarchizacja projektu.}
	\subsection{Moduł 1.}
	\subsection{Moduł 2.}
	\subsection{Moduł 3.}
	\subsection{Moduł 4.}
	
	\section{Pomiary parametrów po ekstrakcji.}
	\subsection{Sposób pomiaru}
	Jak przed ekstrakcją.
	\subsection{Rozmiar układu, liczba tranzystorów.}

	\chapter{Podsumowanie}
	\section{Weryfikacja osiągniętych celów.}
	
	
	\appendix
	%insert chapters	
.
	\begin{thebibliography}{9}
		%%moje z knovela
		\bibitem{integconv} 
		Jespers, Paul G.A.. (2001). 
		\textit{Integrated Converters - D to A and A to D Architectures, Analysis and Simulation.}
		Oxford University Press
		
		\bibitem{cmosanal} 
		Allen, Phillip E. Holberg, Douglas R.. (2012)  
		\textit{CMOS Analog Circuit Design (3rd Edition). }
		Oxford University Press.
		
		\bibitem{vlsidesign} 
		Das, Debaprasad. (2015). .
		\textit{VLSI Design (2nd edition)}
		Oxford University Press.
		%%z opisu tematu od profa allen cmos sie pokrywa z moim wyborem
		\bibitem{plassche} 
		R. Plassche (2001). 
		\textit{Scalone przetworniki analogowo-cyfrowe i cyfrowo-analogowe,}
		WKŁ 2001
	\end{thebibliography}

\end{document}
