\documentclass[10pt,a4paper]{report}
 \usepackage{booktabs}
\usepackage{float}
\usepackage{tabularx}
\usepackage{amsmath}
\usepackage{amsfonts}
\usepackage{amssymb}
\usepackage{graphicx}
\usepackage{amsthm}

\usepackage{polski}
\usepackage[polish]{babel}
\usepackage[utf8]{inputenc}
\usepackage[polish]{babel}

\usepackage{indentfirst}
\usepackage{ifthen}
\usepackage{hyperref}
\hypersetup{
	colorlinks=true, %set true if you want colored links
	linktoc=all,     %set to all if you want both sections and subsections linked
	linkcolor=red,  %choose some color if you want links to stand out
}

\usepackage{tikz}
\def\checkmark{\tikz\fill[scale=0.4](0,.35) -- (.25,0) -- (1,.7) -- (.25,.15) -- cycle;} 
\usepackage{color, colortbl}

\usepackage{pgfplots}       % <-- required in preamble
\usepackage{pgfplotstable}       % <-- required in preamble
%\usepackage{SIunits}        % <-- required in preamble
\pgfplotsset{compat=newest} % <-- optional in preamble


\usepackage[americancurrents]{circuitikz}


\usetikzlibrary{shapes,arrows,calc}

\tikzstyle{block} = [rectangle, draw, text width=3cm, 
text centered, rounded corners, 
minimum height=1.5cm, node distance=2cm]

\tikzstyle{signal} = [rectangle, text width=2.5cm, 
text centered, rounded corners, 
minimum height=1cm, node distance=2cm]

\tikzstyle{line} = [draw, -latex']

\usepackage{geometry}
\geometry{legalpaper, margin=1in}

\theoremstyle{definition}
\newtheorem{notation}{Notacja}[section]

\theoremstyle{definition}
\newtheorem{definition}{Definicja}[section]

\theoremstyle{definition}
\newtheorem{przyklad}{Przykład}[section]

\theoremstyle{definition}
\newtheorem{twierdzenie}{Twierdzenie}[section]

\theoremstyle{definition}
\newtheorem{wniosek}{Wniosek}[section]

\newcommand{\img}[4]{
	\begin{figure}[H]
		\begin{center}
			\includegraphics[width=#1 cm, keepaspectratio=true]{#2}
			\caption{#3}
			\label{#4}
		\end{center}
	\end{figure}
}

\definecolor{Gray}{gray}{0.9}
\setcounter{secnumdepth}{4}

\title{Notatki do pracy inżynierskiej. \\ 8-bitowy przetwornik cyfrowo analogowy w technologii CMOS. \\ STAN : alpha}
\date{2018}
\author{Michał Czyż \\ student, WETI PG}

\begin{document}
	\maketitle
	
	\tableofcontents
	\newpage
	\chapter{Preliminaria.}
	\section{Abstract.}
	{	Goal of this thesis... \textcolor{red}{NA KONIEC UZUPEŁNIĆ.} }

	\section{Abstrakt.}
	{	Prztworniki cyfrowo-analogowe są niezbędną częścią wielu systemów mikroelektronicznych. Celem niniejszej pracy jest zaprojektowanie schematu i topografii masek 8-bitowego przetwornika cyfrowo analogowego w technologii CMOS AMS 180nm. Zaprojektowany przetwornik jest to kaskada dwóch 4-bitowych równoległych konwerterów ze skalowaniem prądu. Wykonano serię symulacji służących do pomiaru parametrów konwertera.}
	
	\section{Podziękowania.}
	{	Dziękujemy wszystkim ... . \textcolor{red}{NA KONIEC UZUPEŁNIĆ} }
	
	\chapter{Wstęp.}
	{	Konwertery sygnałów z postaci cyfrowej na analogową (i odwrotnie) są niezbędną częścią systemów elektronicznych, ponieważ umożliwiają komunikację pomiędzy zewnętrznym, analogowym światem i cyfrowymi rdzeniami układów krzemowych \cite{integconv}. Do przykładowych zastosowań konwerterów danych należą m.in. generowanie sygnału wizyjnego, fonicznego lub sygnałów sterowania np. dla układów radarowych, konwerterów mocy lub miernictwa. }
	

	\section{Systemy konwersji cyfrowo analogowej.}
	\subsection{Twierdzenie Nyquista}
	{	W tym rozdziale dokonano zbioru najistotniejszych pojęć niezbędnych do zrozumienia sposobu działania konwerterów danych i zagadnień związanych z przetwarzaniem sygnałów. }
	
	\begin{twierdzenie}{Niech sygnał $S$ zajmuje pasmo częstotliwości $B =\left(0, f_{max}\right) $. Sygnał $S$ może być przetwarzany bezbłędnie, wtedy i tylko wtedy, gdy częstotliwość próbkowania $f_{sample}$ jest conajmniej dwukrotnie większa od górnego zakresu pasma częstotliwości $f_{max}$. }
	$$
		f_{sample} \ge 2 f_{max}
	$$
	\end{twierdzenie}
	
	\begin{przyklad}{Dany jest konwerter danych, który przetwarza 4 miliony próbek na sekundę ($4MSPS$). Zgodnie z twierdzeniem Nyquista maksymalna częstotliwość przetwarzanego sygnału wynosi:}
		\begin{equation}
			f_{max} = \frac{f_{sample}}{2} = \frac{4M}{2} = 2MHz
		\end{equation}
	\end{przyklad}
	
	\subsection{Filtracja wejściowa}
	{	W przypadku gdy twierdzenie Nyquista nie jest spełnione w widmie sygnału wyjściowego obserwuje sie imaże (zwielokrotnienia) widm, które powodują zniekształcenie sygnału. Z tego powodu do poprawnego funkcjonowania systemu konwersji cyfrowo analogowej przed blokiem konwersji należy umieścić cyfrowy filtr dolnoprzepustowy, którego zadaniem jest ograniczenie pasma sygnału wejściowego w taki sposób, aby spełnione było twierdzenie Nyqsuita. Tego rodzaju filtr jest nazywany filtrem antyaliasowym. }
	
	\begin{przyklad}{Dany jest konwerter danych, który przetwarza $4MSPS$. Filtr antyaliasowy o transmitancji $H(f)$ powinien mieć charakterystykę amplitudową:}
		\[   
		|H(f)| = 
		\begin{cases}
			1 &\quad \text{dla} f \le 2MHz\\
			0 &\quad \text{dla} f>2MHz\\
		\end{cases}
		\]		
	\end{przyklad}
	\begin{figure}[!htb]
	\centering
	\begin{tikzpicture}
	\draw[->] (-4,0) -- (4,0) node[below] {$f$};
	\draw[->] (0,-1) -- (0,2) node[right] {$|H(f)|$};
	\draw[scale=1,domain=-2:2,smooth,variable=\x,blue] plot ({\x},{1});
	\draw[scale=1,domain=-4:-2,smooth,variable=\x,blue] plot ({\x},{0});
	\draw[scale=1,domain=2:4,smooth,variable=\x,blue] plot ({\x},{0});
	\draw[scale=1,domain=0:1,smooth,variable=\y,blue] plot ({-2},{\y});
	\draw[scale=1,domain=0:1,smooth,variable=\y,blue] plot ({2},{\y});
	\end{tikzpicture}
	\end{figure}
		
	\subsection{Filtracja wyjściowa}
	{	W trakcie konwersji wyjście konwertera znajduje się w stanie przejściowym, co wprowadza
	zniekształcenia sygnału wyjściowego. Wynika to zazwyczaj z faktu, że przejściu pomiędzy różnymi słowami cyfrowymi towarzyszy przełączenie jednego lub więcej kluczy. Ochronę przed zniekształceniami sygnału i impulsami szpilkowymi osiąga się dzięki stosowaniu układu próbkująco-pamiętającego i filtru wyjściowego. Ten ostatni popularnie nazywany jest filtrem rekonstrukcyjnym. }
	\begin{przyklad}{Dany jest konwerter danych, który przetwarza $4MSPS$. Filtr rekonstrukcyjny powinien być filtrem dolnoprzepustowym o częstotliwości granicznej równej $2MHz$.}
	\end{przyklad}

	\begin{przyklad}{Dany jest konwerter danych, który przetwarza $4MSPS$. Układ próbkująco-pamiętający musi być szybszy niż... stała czasowa, 1/10zegara?}
	\end{przyklad}

	\subsection{System przetwarzania cyfrowo-analogowego.}
	{	Łącząc informacje przytoczone w poprzednich punktach można stworzyć ogólny diagram systemu przetwarzania cyfrowo-analogowego.
		}
	\begin{figure}[!htb]
	\centering
		\begin{tikzpicture}[auto]
		\node [signal] (in) {Sygnał \\ cyfrowy};
		\node [block, below of = in, node distance = 2cm] (fir) {FDP antyaliasing};
		\node [block, below of = fir, node distance = 2cm] (converter) {Blok \\ konwersji C/A};
		\node [block, below of = converter, node distance = 2 cm](sh){Układ próbkująco-pamiętający S\&H};
		\node [block, below of = sh, node distance = 2 cm](afilter){FDP rekonstrukcja};
		\node [signal, below of = afilter, node distance = 2cm] (out) {Sygnał analogowy};
		
		\draw [->] (in) -- node {} (fir);
		\draw [->] (fir) -- node {} (converter);
		\draw [->] (converter) -- node {} (sh);
		\draw [->] (sh) -- node {} (afilter);
		\draw [->] (afilter) -- node {} (out);
		\end{tikzpicture}
	\caption{Schemat blokowy systemu konwersji sygnału.}
	\end{figure}
	
	
	\section{Przegląd architektur konwerterów C/A. \cite{cmosanal} \cite{plassche}}
	{	Dokonuje się kilku podziałów konwerterów. Ze względu na liczbę przetwarzanych bitów wyróżnia się: \textbf{szeregowe}, czyli takie, które dokonują konwersji słowa cyfrowego bit po bicie oraz \textbf{równoległe}, czyli takie, które dokonują konwersji całego słowa jednocześnie. Jeżeli sygnał wyjściowy przetwornika jest stały w czasie dla ustalonego i podtrzymywanego słowa cyfrowego, to nazywamy taki przetwornik \textbf{statycznym}, w przeciwieństwie do przetworników \textbf{dynamicznych}, których sygnał wyjściowy zanika i wymaga odświeżania. }
	
	{	\textbf{Przetwornik z modulacją szerokości impulsów} dokonuje porównania słowa cyfrowego z liniowo rosnącym cyfrowym słowem odniesienia. Wygenerowany w ten sposób impuls o szerokości zależnej od wartości przetwarzanego słowa poddawany jest filtracji dolnoprzepustowej. Do wad tego rozwiązania należą mała szybkośc przetwarzania i konieczność stosowania filtów o wysokim tłumieniu w paśmie zaporowym. }
	
	{	\textbf{Przetwornik integracyjny z całkowaniem liniowym} również wykorzystuje mechanizm porównywania liczb cyfrowych do wygenerowania sygnału o modulowanej szerkości, który jest poddawany scałkowaniu, a następnie spróbkowaniu przez układ próbkująco pamiętający. Celem zwiększenia szybkości działania układu dokonuje się podzielenia słowa bitowego na część starszą i młodszą, a przetwarzanie obu części odbywa się równolegle. }
	
	{	\textbf{Przetworniki sieciowe ze skalowaniem} dokonują zamiany słowa cyfrowego na napięcia, prądy lub ładunki proporcjonalne (ważone) do wartości tego słowa. Sygnały z poszczególnych gałęzi sieci są sumowane, a sygnał wyjściowy podlega konwersji i/lub kondycjonowaniu do zadanej formy. Udaje się uzyskać rozdzielczość do 10-bitów.}
	
	{	\textbf{Przetwornik z kształtowaniem szumu} wykorzystuje filtry nadpróbkujące, układy kształtowania szumu, 1-bitowy przetwornik C/A i analogowy filtr wyjściowy. Uzyskuje się wysoką precyzję 16-18 bitów. }
	
	{	\textbf{Przetwornik szeregowy z redystrybucją ładunku} wykorzystuje układ dwóch połączonych równolegle identycznych kondensatorów i kilka kluczy. Kondensator wejściowy jest ładowany lub rozładowywany w zależności od wartości kolejnych bitów, a następnie dołączany do kondensatora wyjściowego, co pozwala na dodanie lub odjęcie ładunku, a więc zwiększenie lub zmniejszenie wartości napięcia wyjściowego.
	}
	
	{	\textbf{Przetwornik algorytmiczny potokowy} składa się z sumatorów i bloków opóźnienia, które pozwalają na dodawanie przeskalowanych napięć referencyjnych w kolejnych blokach. \textcolor{red}{przepisać} }

	{	\textbf{Przetwornik algorytmiczny iteracyjny} w weźle układu próbkująco pamiętającego dodaje kolejne bity. \textcolor{red}{przepisać} }

	{ 	\textbf{Przetworniki segmentowe} to przetworniki składające się z połączenia dwóch lub więcej przetworników. Zazwyczaj łączone są przetowrniki o małej liczbie bitów w większe, aby zwiększyć rozdzielczość. }

	{	Na podstawie wymagań należy zaproponować architekturę urządzenia na podstawie analizy obecnego stanu wiedzy technicznej lub przeprowadzić zgrubne (często ręczne) obliczenia, potwierdzające spełnienie wymagań. Po zapoznaniu się z literaturą wybór padł na konwerter skalujący prąd. Najistotniejsze motywacje tego wyboru to prostota schematu, wysoka szybkość działania \cite{cmosanal} oraz fakt, że poprzednie prace uzyskały do 10 bitów rozdzielczości \cite{plassche}.	Autorzy \cite{cmosanal} jako główne wady podają duży rozmiar oraz duży rozstęp wielkości elementów.}
	\section{Równoległy przetwornik skalujący prąd.}
	
	{	Blokowy schemat konwertera skalującego prąd znajduje sie na rysunku \ref{currentscale}. Zasada działania jest następująca: słowo cyfrowe jest sygnałem sterującym sieć, która może wytwarzać prądy o różnych wartościach. Typowo, przypisuje się prądom wagi dwójkowe, a kolejne bity słowa cyfrowego decydują o włączeniu/wyłączeniu gałęzi prądowej, co oznacza, że po zsumowaniu prądów wyjściowych otrzymujemy prąd o wartości  bezwzględnej odpowiadającej wartości zakodowanej w słowie cyfrowym. W zależności od wymagań na sygnał wyjściowy, dokonuje się konwersji prądu wyjściowego na napięcie. }


\begin{figure}[!htb]
	\centering
	\begin{tikzpicture}[auto]
	\node [signal, node distance = 2cm] (digint) {N-bitowe słowo cyfrowe};
	\node [block, below of = digint, node distance = 2cm] (network) {Sieć skalująca prąd};
	\node [block, left of = network, node distance = 4cm] (iref1) {Źródło prądu referencyjnego};
	\node [block, below of = network, node distance = 2cm] (conv) {Konwerter prąd/napięcie};
	\node [block, left of = conv, node distance = 4cm] (vref1) {Źródło napięcia referencyjnego};
	\node [signal, below of = conv, node distance = 2cm] (out) {Sygnał \\ analogowy};
	
	\draw [->] (iref1) -- node {$I_{ref}$} (network);
	\draw [->] (digint) -- node {$B=\{b_0, b_1, \dots b_{N-1} \}$} (network);
	\draw [->] (vref1) -- node {$V_{ref}$} (conv);
	\draw [->] (network) -- node {$I_{out}=f(I_{ref},B)$} (conv);
	\draw [->] (conv) -- node {$V_{out}=g(V_{ref},I_{out})$} (out);
	\end{tikzpicture}
	\caption{Schemat blokowy konwertera skalującego prąd.}
\end{figure}

	\section{Segmentowy równoległy przetwornik skalujący prąd.}
{	
	\begin{equation}
	I_{out} = \sum_{i=0}^{N-1} b_i I_{i} = I_{ref} \sum_{i=0}^{N-1} b_i 2^{i}
	\end{equation}
	Zauważmy, że sumując wyjścia dwóch przetworników otrzymujemy w ogolności
	\begin{equation}
	I_{out,MN} = I_{out,M} + I_{out,N}
	= I_{ref,M} \sum_{i=0}^{M-1} b_i 2^{i} + I_{ref,N} \sum_{i=0}^{N-1} b_i 2^{i}
	\end{equation}
	W szczególności, gdy zachodzi $I_{ref,N} = 2^M I_{ref} $:
	\begin{equation}
	I_{out,MN} = I_{ref,M}\left(\sum_{i=0}^{M-1} b_i 2^{i} + 2^M \sum_{i=0}^{N-1} b_i 2^{i}\right) 
	\end{equation}
	\begin{equation}
	I_{out,MN} = I_{ref,M}\left(\sum_{i=0}^{M-1} b_i 2^{i} + \sum_{i=M}^{M+N-1} b_i 2^{i}\right) 
	\end{equation}
	\begin{equation}
	I_{out,MN} = I_{ref,M} \sum_{i=0}^{M+N-1} b_i 2^{i}
	\end{equation}
	Jest to równoważne wyrażeniu na prąd wyjściowy konwertera o $M+N$ bitach rozdzielczości.
	\begin{przyklad}{Niech dany będzie M=4 bitowy konwerter o prądzie referencyjnym $I_{ref,M}$. Sumując wyjście tego przetwornika z przetwornikiem o N=4 bitach i prądzie referencyjnym $ I_{ref,N} = 2^4 \cdot I_{ref,M} $, tworzy się przetwornik o $M+N=8$ bitach.}
	\end{przyklad}	
}

	\section{Wymagania.}
{	Projekt układu scalonego rozpoczyna się od sformułowania wymagań (specyfikacji) urządzenia lub systemu. Sformułowano następujące wymagania na przetwornik:
	\begin{itemize}
		\item zasilany ze źródła napięcia stałego o wartości 1.8V,
		\item zaprojektowany w technologii CMOS AMS 180nm,
		\item osiąga szybkość konwersji powyżej 4 milionów próbek na sekundę (MSPS),
		\item ma 8-bitów rozdzielczości,
		\item sygnał wyjściowy to sygnał napięciowy,
		\item obciążenie ma charakter pojemnościowy.
\end{itemize} }

\section{Uwagi do przemyślenia.}
{	Typową praktyką jest doposażenie konwertera w interfejs cyfrowy, aby ułatwić integrację przetwornika z pozostałymi blokami systemu. W przypadku projektowania przetworników, które samodzielnie tworzą układ scalony umieszcza się także standardowe interfejsy cyfrowe, np. SPI lub I2C. Ponadto, na rynku dostępne są konwertery wyposażone w rejestr kontrolny, który umożliwia wykorzystanie dodatkowych funkcjonalności np. programowania wzmocnienia sygnału wyjściowego, autotestowania lub autokalibracji. }

{	Poprawnie zaprojektowane układy scalone charakteryzują się niewrażliwością na 3 istotne czynniki:
	\begin{itemize}
		\item rozrzut technologiczny (precyzja wykonania, niedopasowanie) parametrów elementów elektronicznych,
		\item zmiana temperatury,
		\item odchyłki napięcia zasilania od wartości nominalnej.
	\end{itemize}
}
{	Kluczowym dla poprawnego działania konwertera skalowanego jest stabilne źródło napięcia odniesienia. Układy wytwarzane w technologii CMOS są wrażliwe na rorzut technologiczny parametrów elementów elektronicznych, zmieniają się ich właściwości pod wpływem zmian temperatury lub zmian wartości napięcia zasilania. Aby zapewnić stabilność napięcia odniesienia stosuje się układ bandgap. Na podstawie literatury w tej technologii można uzyskać źródła napięcia odniesienia niezależne od temperatury o wartości około $V_{REF}=1.1V$. Taką wartość napięcia referencyjnego przyjmiemy dla pozostałych bloków. }


	

	
	\section{Parametry statyczne przetworników C/A.}
	{	Sygnałem wejściowym dla N bitowego przetwornika cyfrowo analogowego jest słowo cyfrowe \\ $\mathcal{B}=\{b_{N-1},b_{N-2},...,b_0\}$. Bit $b_{N-1}$ nosi miano najstarszego bitu (MSB - Most Siginificant Bit), a bit $b_0$ najmłodszego (LSB - Least Siginificant Bit). Sygnał wyjściowy to analogowe napięcie lub prąd $\mathcal{S}$ przeskalowany przez sygnał referencyjny $V_{ref}$. Zależność pomiędzy sygnałem wyjściowym i wejściowym to wtedy:
	\begin{equation}
	\mathcal{S} = V_{ref}\mathcal{B} = V_{ref} {\sum_{k=0}^{N-1} b_k2^k}
	\end{equation}	}
	
	\begin{notation}{Sygnał wyjściowy należy do zakresu:}
		$$
			S_{out} \in \langle S_{out_{min}} ;  S_{out_{max}} \rangle
		$$
	\end{notation}

	\begin{definition}{Zakres sygnału wyjściowego FS to zakres wartości, które może przyjmować sygnał wyjściowy przetwornika.}
		\begin{equation}
			FS = S_{out_{max}} - S_{out_{min}}
		\end{equation}
	\end{definition}

	\begin{definition}{Rozdzielczość R określa minimalną zmianę sygnału wyjściowego dla kolejnych słów cyfrowych. Odpowiada stosunkowi pełnego zakresu napięcia wyjściowego do liczby poziomów (różnych słów cyfrowych). }
		\begin{equation}
			R = \frac{FS}{2^{N}-1}		
		\end{equation}
	\end{definition}


	\begin{definition}{Dokładność bezwzględna}
		
	\end{definition}
	
	\begin{definition}{Dokładność względna $\Delta$ to odchylenie wartości sygnału wyjściowego w stanie ustalonym od teoretycznej prostej wyznaczonej przez pełny zakres przetwarzania. }
		
	\end{definition}
	
	\begin{definition}{Sygnał wyjściowy przetwornika \textbf{monotonicznego} nie zmniejsza się dla wzrostu wejściowego słowa cyfrowego. }
		
	\end{definition}


	\begin{definition}{Przesunięcie zera to wartość sygnału wyjściowego dla sygnału wejściowego o wartości zero. }
	
	\end{definition}


	\begin{definition}{Nieliniowość całkowa INL to wartość bezwzględna sumy błędów nieliniowości dodatnich (lub ujemnych).}
		
	\end{definition}


	\begin{definition}{Nieliniowość różniczkowa DNL to zmiana sygnału wyjściowego w stosunku do LSB przy przejściu o jedno słowo cyfrowe obliczone dla każdego przejścia osobno. }

	\end{definition}	

	\section{Parametry dynamiczne przetworników C/A.}
	{ \textbf{\textcolor{red}{Wymienić resztę.}}
	}

	{
		\begin{itemize}
			\item Stosunek sygnału do szumu.
			\item Maksymalna częstotliwośc próbkowania,
			\item PSRR.
		\end{itemize} 
	}

	\chapter{Technologia AMS 180nm.}
	
	\section{Dostępne urządzenia.}
	{	W projekcie wykorzystano następujące tranzystory dostarczone przez producenta technologii:
		\begin{itemize}
			\item NFET,
			\item NFETX,
			\item PFET,
			\item PFETX
		\end{itemize}
   }
	\section{Model kwadratowy tranzystora LV-FET z kanałem indukowanym do obliczeń ręcznych.}
	{ 	W pracy wykorzystywane są niskonapięciowe tranzystory polowe z kanałami indukowanymi typu n i p. Do obliczęń ręcznych wykorzystano uproszczony model kwadratowy. Wyróżnia się w nim obszary pracy tranzystora: odcięcia, triodowy (inaczej: nienasycenia, omowy), pentodowy (nasycenia). }
	
	{	Obszar odcięcia tranzystora NFET:
		\begin{equation}
			V_{GS} \leq V_{TH,n}
		\end{equation}
		
		\begin{equation}
			I_{DS} \eqsim 0
		\end{equation}
	}
	
	{	Obszar nienasycenia tranzystora NFET:
		\begin{equation}
			V_{GS} > V_{TH,n}
		\end{equation}
		
		\begin{equation}
			V_{DS} \leq V_{GS} - V_{TH,n}
		\end{equation}
		
		\begin{equation}
			I_{DS} = \mu_n C_{ox} \frac{W}{L} \left( \left(V_{GS} - V_{TH,n}\right) V_{DS} - \frac{V_{DS}^2}{2}  \right)
		\end{equation}
	}

	{	Obszar nasycenia tranzystora NFET:
		\begin{equation}
			V_{GS} > V_{TH,n}
		\end{equation}
		
		\begin{equation}
			V_{DS} > V_{GS} - V_{TH,n}
		\end{equation}
		
		\begin{equation}
			I_{DS} = \frac{1}{2} \mu_n C_{ox} \frac{W}{L} \left( V_{GS} - V_{TH,n} \right)^2
		\end{equation}	
	}

	{	Obszar odcięcia tranzystora PFET:
		\begin{equation}
		V_{SG} \leq \left|V_{TH,p}\right|
		\end{equation}
		
		\begin{equation}
		I_{SD} \eqsim 0
		\end{equation}
	}
	
	{	Obszar nienasycenia tranzystora PFET:
		\begin{equation}
		V_{SG} > \left|V_{TH,p}\right|
		\end{equation}
		
		\begin{equation}
		V_{SD} \leq V_{SG} - \left|V_{TH,p}\right|
		\end{equation}
		
		\begin{equation}
		I_{DS} = \mu_p C_{ox} \frac{W}{L} \left( \left(V_{SG} - \left|V_{TH,p}\right| \right) V_{SD} - \frac{V_{SD}^2}{2}  \right)
		\end{equation}
	}
	
	{	Obszar nasycenia tranzystora PFET:
		\begin{equation}
		V_{SG} > \left|V_{TH,p}\right|
		\end{equation}
		
		\begin{equation}
		V_{SD} > V_{SG} - \left|V_{TH,p}\right|
		\end{equation}
		
		\begin{equation}
		I_{DS} = \frac{1}{2} \mu_p C_{ox} \frac{W}{L} \left( V_{SG} - \left|V_{TH,p}\right| \right)^2
		\end{equation}	
	}

	{	Ponadto wprowadzono oznaczenia:
	
		$$
			K_i = \mu_i C_{ox}
		$$
	
		$$
			\beta_i = \mu_i C_{ox} \frac{W}{L} = K_i \frac{W}{L}
		$$
	}

	{	Z dokumentacji wyciągnięto następujące wartości stałych procesu:
		$$
			V_{TH,n} = 0.355V
		$$
		
		$$
			K_n = 274 \frac{\mu A}{V^2}
		$$

		$$
			V_{TH,p} = -0.405V
		$$
	
		$$
			K_p = 56 \frac{\mu A}{V^2}
		$$	
	}
	\section{Zakres temperatur.}
	{	Proces został zakwalifikowany do zakresu temperatur złącza $-40^\circ C \leq T \leq 150^\circ C$. }

	\chapter{Projekt wstępny.}


	\section{Zaproponowana architektura.}
	{	Zaproponowano przetwornik równoległy, skalujący prąd oparty o sieć luster prądowych, złożony z dwóch kaskadowo połączonych przetworników 4-bitowych. Prąd referencyjny $I_{ref}$ przepływający przez tranzystor $M_0$ jest replikowany przez tranzystory $M_1, M_2, \dots M_N$ ze współczynnikami binarnymi $1,2,\dots, 2^{N-1}$. Wzmacniacze $A_1$ i $A_2$ służą stabilizacji napięć dren źródło, niezbędne dla prawidłowej pracy luster prądowych. Tranzystory $K_1, K_2, \dots K_N$ służą za klucze. Prądy gałęzi są sumowane w węźle $A_1.-$ i konwertowane na napięcie na rezystorze $R$. W dalszej części pracy przeanalizowano 3 istotne fragmenty tego układu: lustra prądowe, klucze i wzmacniacze operacyjne. Wyprowadzono podstawowe zależności, które decydują o wymaganiach dotyczących wymienionych elementów.}

	\begin{figure}[!htb]
	\centering
	\begin{circuitikz}[scale = 0.6]
%		\draw [gray, color=gray] (0,-1) grid (20,10);
		\draw [color=black, thick]
		% Devices
		(2, 2) node[op amp, rotate = 270] (opamp) {$A_1$}
		(18, 7) node[op amp] (opamp2) {$A_2$}
		(4, 1) node[nmos, rotate =180, label={ [centered,xshift=-10, yshift = 10] {$M_0$} } ](m0){};
		\foreach \x in {1,...,4}
		{
			\pgfmathsetmacro{\offsetx}{int(3*\x+5)};
			\draw [color=black, thick]
			(\offsetx,1) node[nmos, label={ [centered,xshift=10, yshift = 10] {$M_\x$} } ](m\x){}
			(\offsetx,3) node[nmos, label={ [centered,xshift=10, yshift = 10] {$K_\x$} } ](k\x){}
			(k\x.D) to [short] (\offsetx,5)
			;
		}
		\draw[color=black, thick]
		% Ports
		(opamp.+) to [short, -o](1.2,5)
		(opamp.-) to [short](2.7,4)
		(opamp.out) to [short](2,0)
		(2,0) to [short](17,0)
		(5,1) to [short](16,1)
		(2.66,4) to [short](4,4)
		(4,4) to [short](4,2)
		(1.5,6) node[]{\large{\textbf{$V_{ref}$}}}
		(8,5) to [short](17,5)
		(12,5) to [short](12,7.7)
		(12,7.7) to[short](opamp2.-)
		(15,6.2) to[short,o-](opamp2.+)
		(14,6.2) node[]{\large{\textbf{$V_{ref}$}}}
		(14,7.7) to [short] (14,10)
		(14,10) to[R=$R$] (20,10)
		(20,10) to[short](20,7)
		(20,7) to[short](opamp2.out)
		(4,9) to [I=${I_{ref}}$] (4,4)
		;
		
	\end{circuitikz}
	\caption{4-bitowy przetwornik C/A.}
\end{figure}

	\section{Badanie luster prądowych.}
	{	W pierwszej kolejności wykonano symulacje 4 luster prądowych stabilizowanych idealnymi wzmacniaczami operacyjnymi bez kluczy, celem wyznaczenia minimalnych rozmiarów tranzystorów  $M_1, M_2, M_3, M_4$. Konwerter będzie działał poprawnie, gdy błąd replikacji prądów będzie mniejszy niż połowa LSB. Błąd obliczano:}
	\begin{equation}
		\delta I = \frac{I_{out,measure} - I_{out,ideal}}{I_{ref}} = \frac{I_{out,measure} - 15 I_{ref}}{I_{ref}}
	\end{equation}
	\subsection{Wpływ prądu referenyjnego na błąd replikacji prądu.}
	{ Wykonano symulację DC dla następujących parametrów:
	\begin{itemize}
		\item $A_d = 100M$, $V_{off} = 0V$
		\item $I_{ref} = 100nA$, $V_{ref} = 900mV$
		\item $L_0 = 1.8\mu$, $W_0 = 3.6\mu$
	\end{itemize} }
	
	\begin{figure}[!htb]
	\centering
	\begin{tikzpicture}
	\begin{axis}[
	width=0.6\linewidth, % Scale the plot to \linewidth
	grid=major, % Display a grid
	grid style={dashed,gray!30}, % Set the style
	xlabel=$ I_{ref} $, % Set the labels
	ylabel=$ \delta I$ ,
	legend style={at={(0.5,-0.2)},anchor=north}, % Put the legend below the plot
	x tick label style={anchor=north} % Display labels sideways
	]
	\addplot[only marks, mark=star, blue] table [x={I}, y={error}, col sep=comma]{\detokenize{1_sweep_iref.csv}};
	\end{axis}
	\end{tikzpicture}
	\caption{Błąd replikacji prądu wyrażony w LSB.}
	\end{figure}
	
	\begin{wniosek}{ Błąd replikacji jest odwrotnie proporcjonalny do prądu referencyjnego.}
	\end{wniosek}

	\subsection{Wpływ napięcia referenyjnego na błąd replikacji prądu.}
		{ Wykonano symulację DC dla następujących parametrów:
		\begin{itemize}
		\item $A_d = 100M$, $V_{off} = 0V$
		\item $I_{ref} = 100nA$, $V_{ref} = 900mV$
		\item $L_0 = 1.8\mu$, $W_0 = 3.6\mu$
	\end{itemize} }
	
	\begin{figure}[!htb]
		\centering
		\begin{tikzpicture}
		\begin{axis}[
		width=0.6\linewidth, % Scale the plot to \linewidth
		grid=major, % Display a grid
		grid style={dashed,gray!30}, % Set the style
		xlabel=$ V_{ref} $, % Set the labels
		ylabel=$ \delta I$ ,
		legend style={at={(0.5,-0.2)},anchor=north}, % Put the legend below the plot
		x tick label style={anchor=north} % Display labels sideways
		]
		\addplot[only marks, mark=star, blue] table [x={I}, y={error}, col sep=comma]{\detokenize{2_sweep_vref.csv}};
		\end{axis}
		\end{tikzpicture}
		\caption{Błąd replikacji prądu wyrażony w LSB.}
	\end{figure}
	
	\begin{wniosek}{ Błąd replikacji jest słabo zależny od napięcia referencyjnego.}
	\end{wniosek}
	
	\subsection{Wpływ szerokości tranzystorów na błąd replikacji prądu.}
	{ Wykonano symulację DC dla następujących parametrów:
		\begin{itemize}
			\item $A_d = 100M$, $V_{off} = 0V$
			\item $I_{ref} = 100nA$, $V_{ref} = 900mV$
			\item $L_0 = 1\mu$, $W_0 = 1\mu$
	\end{itemize} }
		\begin{figure}[!htb]
		\centering
		\begin{tikzpicture}
		\begin{axis}[
		width=0.6\linewidth, % Scale the plot to \linewidth
		grid=major, % Display a grid
		grid style={dashed,gray!30}, % Set the style
		xlabel=$ I_{ref} $, % Set the labels
		ylabel=$ \delta I$,
		legend style={at={(0.5,-0.2)},anchor=north}, % Put the legend below the plot
		x tick label style={anchor=north} % Display labels sideways
		]
		\addplot[only marks, mark=star, blue] table [x={error (W0=1e-06) X}, y={error (W0=1e-06) Y}, col sep=comma]{\detokenize{3_step_W.csv}};
		\addplot[only marks, mark=star, red] table [x={error (W0=1.066667e-05) X}, y={error (W0=1.066667e-05) Y}, col sep=comma]{\detokenize{3_step_W.csv}};
		\addplot[only marks, mark=star, black] table [x={error (W0=2.033333e-05) X}, y={error (W0=2.033333e-05) Y}, col sep=comma]{\detokenize{3_step_W.csv}};
		\addplot[only marks, mark=star, green] table [x={error (W0=3e-05) X}, y={error (W0=3e-05) Y}, col sep=comma]{\detokenize{3_step_W.csv}};
		\end{axis}
		\end{tikzpicture}
		\caption{Błąd replikacji prądu wyrażony w LSB dla różnych szerokości.}
	\end{figure}

	
	\begin{wniosek}{ Błąd replikacji jest mniejszy dla tranzystorów o mniejszej szerokości.}
	\end{wniosek}
	
	\subsection{Wpływ długości tranzystorów na błąd replikacji prądu.}
	{ Wykonano symulację DC dla następujących parametrów:
		\begin{itemize}
			\item $A_d = 100M$, $V_{off} = 0V$
			\item $I_{ref} = 100nA$, $V_{ref} = 900mV$
			\item $L_0 = 1\mu$, $W_0 = 1\mu$
	\end{itemize} }
	
	
	\begin{figure}[!htb]
	\centering
	\begin{tikzpicture}
	\begin{axis}[
	width=0.6\linewidth, % Scale the plot to \linewidth
	grid=major, % Display a grid
	grid style={dashed,gray!30}, % Set the style
	xlabel=$ I_{ref} $, % Set the labels
	ylabel=$ \delta I$,
	legend style={at={(0.5,-0.2)},anchor=north}, % Put the legend below the plot
	x tick label style={anchor=north} % Display labels sideways
	]
	\addplot[only marks, mark=star, blue] table [x={error (L0=1e-06) X}, y={error (L0=1e-06) Y}, col sep=comma]{\detokenize{4_step_L.csv}};
	\addplot[only marks, mark=star, red] table [x={error (L0=5.666667e-06) X}, y={error (L0=5.666667e-06) Y}, col sep=comma]{\detokenize{4_step_L.csv}};
	\addplot[only marks, mark=star, black] table [x={error (L0=1.033333e-05) X}, y={error (L0=1.033333e-05) Y}, col sep=comma]{\detokenize{4_step_L.csv}};
	\addplot[only marks, mark=star, green] table [x={error (L0=1.5e-05) X}, y={error (L0=1.5e-05) Y}, col sep=comma]{\detokenize{4_step_L.csv}};
	\end{axis}
	\end{tikzpicture}
	\caption{Błąd replikacji prądu wyrażony w LSB dla różnych długości.}
\end{figure}

	
	\begin{wniosek}{ Błąd replikacji jest mniejszy dla tranzystorów o większej długości.}
	\end{wniosek}

	\subsection{Wstępny wybór L i W.}
	
		{ Wykonano symulację DC dla następujących parametrów:
		\begin{itemize}
			\item $A_d = 100M$, $V_{off} = 0V$
			\item $I_{ref} = 100nA$, $V_{ref} = 900mV$
			\item $L_0 = 10.8\mu$, $W_0 = 1\mu$
	\end{itemize} }
	
	
	\begin{figure}[!htb]
		\centering
		\begin{tikzpicture}
		\begin{axis}[
		width=0.6\linewidth, % Scale the plot to \linewidth
		grid=major, % Display a grid
		grid style={dashed,gray!30}, % Set the style
		xlabel=$ I_{ref} $, % Set the labels
		ylabel=$ \delta I$,
		legend style={at={(0.5,-0.2)},anchor=north}, % Put the legend below the plot
		x tick label style={anchor=north} % Display labels sideways
		]
		\addplot[only marks, mark=star, blue] table [x={I}, y={error}, col sep=comma]{\detokenize{5_final_LW.csv}};
		\end{axis}
		\end{tikzpicture}
		\caption{Błąd replikacji prądu wyrażony w LSB dla wybranego W/L.}
	\end{figure}
	
	
	\begin{wniosek}{ Dla całego zakresu prądu błąd jest poniżej połowy LSB.}
	\end{wniosek}
	
	\subsection{Sizing klucza.}

	{ Wykonano symulację DC dla następujących parametrów:
		\begin{itemize}
			\item $A_d = 100M$, $V_{off} = 0V$
			\item $I_{ref} = 100nA$, $V_{ref} = 900mV$
			\item $L_0 = 10.8\mu$, $W_0 = 1\mu$
			\item $W_K = 1\mu$, $L_K = 0.18\mu$
	\end{itemize} }
	
	
	\begin{figure}[!htb]
		\centering
		\begin{tikzpicture}
		\begin{axis}[
		width=0.6\linewidth, % Scale the plot to \linewidth
		grid=major, % Display a grid
		grid style={dashed,gray!30}, % Set the style
		xlabel=$ I_{ref} $, % Set the labels
		ylabel=$ \delta I$,
		legend style={at={(0.5,-0.2)},anchor=north}, % Put the legend below the plot
		x tick label style={anchor=north} % Display labels sideways
		]
		\addplot[only marks, mark=star, blue] table [x={error (WK=5e-07) X}, y={error (WK=5e-07) Y}, col sep=comma]{\detokenize{6_step_WK.csv}};
		\addplot[only marks, mark=star, red] table [x={error (WK=7.5e-07) X}, y={error (WK=7.5e-07) Y}, col sep=comma]{\detokenize{6_step_WK.csv}};
		\addplot[only marks, mark=star, black] table [x={error (WK=1e-06) X}, y={error (WK=1e-06) Y}, col sep=comma]{\detokenize{6_step_WK.csv}};
		\end{axis}
		\end{tikzpicture}
		\caption{Błąd replikacji prądu wyrażony w LSB dla wybranego W/L.}
	\end{figure}
	
	\subsection{Symulacja kaskady x2}
	{ Wykonano symulację DC dla następujących parametrów:
		\begin{itemize}
			\item $A_d = 100M$, $V_{off} = 0V$
			\item $I_{ref} = 100nA$, $V_{ref} = 900mV$
			\item $L_0 = 10.8\mu$, $W_0 = 1\mu$
			\item $W_K = 1\mu$, $L_K = 0.18\mu$
	\end{itemize} }
	
	
	\begin{figure}[!htb]
		\centering
		\begin{tikzpicture}
		\begin{axis}[
		width=0.6\linewidth, % Scale the plot to \linewidth
		grid=major, % Display a grid
		grid style={dashed,gray!30}, % Set the style
		xlabel=$ I_{ref} $, % Set the labels
		ylabel=$ \delta I$,
		legend style={at={(0.5,-0.2)},anchor=north}, % Put the legend below the plot
		x tick label style={anchor=north} % Display labels sideways
		]
		\addplot[only marks, mark=star, blue] table [x={I}, y={error}, col sep=comma]{\detokenize{7_sweep_iref_double.csv}};
		\end{axis}
		\end{tikzpicture}
		\caption{Błąd replikacji prądu wyrażony w LSB dla wybranego W/L. Czy to jest dobre .csv powtorzyć.}
	\end{figure}
	
	\begin{wniosek}{}
		\begin{equation}
			\delta I_{cascade}(I_{ref}) >> \delta I_{single}(I_{ref}) + \delta I_{single}(16*I_{ref}) 
		\end{equation}
	\end{wniosek}

	\subsection{Zwiększenie L.}
	
	\subsection{Komentarz output range of amplifier.}
	
	\subsection{Powierzchnia konwertera.}
	{	Na tym etapie warto zauważyć, że wyłącznie powierzchnia tranzystorów w lustrze prądowym staje się znacząca. Zakładając optymistycznie, że rozmiar tranzystora to wyłącznie powierzchnia kanału $A=W*L$, powierzchnia luster prądowych wynosi:
	\begin{equation}
		A = 2 \left( W_0 \cdot L_0 + \sum_{i=0}^{3}2^i W_0 \cdot L_0 \right) = 32 W_0 \cdot L_0
	\end{equation}
	Jeżeli ta powierzchnia okazuje się nieakceptowalna ze względu na cenę lub dostępną wolną powierzchnią w większym systemie, należy rozważyć inne architektury.
	}
	
	\section{Projekt wzmacniacza operacyjnego}
	{	Projekt przetwornika wymaga przygotowania wzmacniaczy operacyjnych. Poniżej przedstawiono projekt wzmacniacza ogólnego przeznaczenia. Stopień wejściowy to para różnicowa obciążona aktywnie. Drugi stopień to inwerter.}
	{	Plan testowania wzmacniacza operacyjnego:
		\begin{enumerate}
			\item Wyznacz napięcie niezrównoważenia $V_{off}$.
				\subitem Schemat do pomiaru: 
			\item Wyznacz ICMR.
				\subitem Dla skrajnych ICMR wszystkie tranzystory są w nasyceniu.
			\item Chtyka częstotliwościowa - $A_v(0)$, $f_{pole}$, $\phi_{margin}$
			\item PSRR, CMRR
		\end{enumerate}
	}

	\subsection{Konwersja prądu na napięcie przy pomocy pojedynczego tranzystora.}
	{	Tranzystor MOSFET połączony jak na \ref{itov} może być traktowany jako przetwornik prądu na napięcie. 
	}

	\begin{figure}[h!]
	\begin{center}
		\begin{circuitikz}
			\draw [color=black, thick]
			%Ground
			(0,0) node[ground]{} 
			(0,0) to [I=${I_{ref}}$] (0,2)
			(2,0) node[ground]{} 
			(2,1) node[nmos, rotate =180, label={ [centered,xshift=-10, yshift = 10] {$M_0$} } ](m0){}
			(0,2) to[short] (2,2)
			(2,0) to[short] (m0.D)
			(2,2) to[] (m0.S)
			(2,2) to[short,*-] (3,2)
			(3,2) to[short,-*] (3,1)
			%Ground line
			(m0.G) to[short,-o] (4,1)
			(4.5,1) node[]{\large{\textbf{$V_{out}$}}}
			;
		\end{circuitikz}
		\caption{Konwersja prądu na napięcie.}
		\label{itov}
	\end{center}	
	\end{figure}

	
	{	Zakładając, że tranzystor pracuje w obszarze aktywnym, prąd drenu jest równy:
		\begin{equation}
		I_{REF} = K' \frac{W}{L}\left( V_{GS}-V_{TH,n}\right)^2
		\end{equation}
	Zaniedbując efekt modulacji kanału i uwzględniając, że źródło tranzystora jest na poziomie masy sygnałowej, napięcie wyjściowy może zostać wyliczone jako:
		\begin{equation} \label{eq_v_out_mosfet}
		V_{OUT} = V_{D} = V_{G} = \sqrt{\frac{I_{REF}}{K'}\frac{L}{W}} + V_{TH,n}
		\end{equation}
	}

	\subsection{Konwersja prądu na napięcie przy pomocy tranzystora i wzmacniacza operacyjnego.}
	
	\begin{figure}
		\centering
		\begin{circuitikz}
			\draw [color=black, thick]
			%Ground
			
			(0, 1) node[op amp, rotate =270] (opamp) {$A_1$}
			(opamp.+) node[ground, rotate = 180] {}
			
			(opamp.-) to [short] (0.5,2.5)
			(2,5) to [I=${I_{ref}}$] (2,3)
			(2,5) node[ground, rotate = 180 ]{}
			(2,1) node[nmos, rotate =180, label={ [centered,xshift=-10, yshift = 10] {$M_0$} } ](m0){}
			(0.5,2.5) to[short] (2,2.5)
			(2,0) to[short] (m0.D)
			(opamp.out) to [short](0,-0.5)
			(2,3) to [] (m0.S)
			(0,-0.5) to [short] (2,-0.5)
			(m0.D) to [] (2,-0.5)
			(2,2) to[] (m0.S)
			(2,2.5) to[short,*-] (3,2.5)
			(3,2.5) to[short,-*] (3,1)
			%Ground line
			(m0.G) to[short,-o] (4,1)
			(4.5,1) node[]{\large{\textbf{$V_{out}$}}}
			;
		\end{circuitikz}
	\caption{Konwersja z wzmacniaczem.}
	\end{figure}
		
	
		
	
	{ Wzmacniacz operacyjny w pętli sprzężenia zwrotnego wymusi potencjał masy na wejściu minus. Biorąc dane z modelu do obliczeń ręcznych i dla rysunku powyżej, z \ref{eq_cutoff} otrzymujemy nierówność:
		$$
		0 - V_s > V_{th,n} = 0.4
		$$
		Źródło prądowe wymusza prąd $I_{ref}$, więc:
		$$
		I_{ref} = K \frac{W}{L}\left( V_s-V_{th,n}\right)^2
		$$
		Zakładając jednostkowy stosunek szerkości do wyskości i podstawiając dane otrzymujemy dla $I_{ref} = 10\mu A$:
		$$
		V_s = -0.839V
		$$
	}


	\subsection{Analiza lustra prądowego z kluczem z pojedynczego nMOSFET.}
	{ Projektując lustro prądowe, należy zadbać, aby tranzystory były w obszarze aktywnym, czyli:
		\begin{equation} \label{eq_cutoff}
			V_{GS} > V_{TH,n}
		\end{equation}
		oraz
		\begin{equation} \label{eq_saturation}
			V_{DS} \ge V_{GS} - V_{TG,n}
		\end{equation}
	Prąd 
		\begin{equation}
			I_{D} = \beta \left( V_{GS} - V_{TH,n} \right)^2 \left(1+\lambda V_{DS} \right)
		\end{equation}
	Zakładając, że wzmacniacz ustali $V_{GS}$ obu tranzystorów w lustrze na ten sam poziom:
		\begin{equation}
			\frac{I_o}{I_{ref}} = \frac{\left(1+\lambda V_{DS,1} \right)}{\left(1+\lambda V_{GS,0} \right)}
		\end{equation}
	}

	\subsection{Źródło napięcia referencyjnego zbudowane z dwóch tranzystorów.}
	obwód generacji napięcia odniesienia jak z rysunku...
	Ze względu na połączenie drenu i bramki, tranzystor może pracować tylko w nasyceniu lub odcięciu. Zakładając, że oba tranzystory
	są w nasyceniu i oznaczając napięcie drenu jako $V_{out}$ układamy równania dla nMOSFET:
	\begin{equation}
	V_{out} > V_{th,n}
	\end{equation}
	
	\begin{equation}
	V_{out} > V_{out} - V_{th,n}
	\end{equation}
	
	i pMOSFET:
	
	\begin{equation}
	V_{dd} - V_{out} > \left|V_{th,p}\right|
	\end{equation}
	
	\begin{equation}
	V_{dd} - V_{out} > V_{dd} - V_{out} - \left|V_{th,p}\right|
	\end{equation}
	
	Po prostych przekształceniach:
	\begin{equation}
	V_{out} > V_{th,n}
	\end{equation}
	
	\begin{equation}
	0 > - V_{th,n}
	\end{equation}
	
	i pMOSFET:
	
	\begin{equation}
	V_{dd} - V_{out} > \left|V_{th,p}\right|
	\end{equation}
	
	\begin{equation}
	0 > - \left|V_{th,p}\right|
	\end{equation}
	
	
	Ponadto, prądy muszą być równe:
	\begin{equation}
	I_{d,n} = I_{d,p}
	\end{equation}
	
	W nasyceniu:
	\begin{equation}
	\beta_n \left(V_{out} - V_{th,n}\right)^2 = \beta_p \left(V_{dd} - V_{out} - \left|V_{th,p}\right|\right)^2
	\end{equation}
	Obie strony są zawsze dodatnie, więc możemy pierwiastkować:
	\begin{equation}
	\left(V_{out} - V_{th,n}\right) = \sqrt{\frac{\beta_p}{\beta_n}}\left(V_{dd} - V_{out} - \left|V_{th,p}\right|\right)
	\end{equation}
	
	\begin{equation}
	V_{out}\left(1+\sqrt{\frac{\beta_p}{\beta_n}}\right) = \sqrt{\frac{\beta_p}{\beta_n}}\left(V_{dd} -\left|V_{th,p}\right|\right)+V_{th,n}
	\end{equation}
	
	\begin{equation}
	V_{out} =\frac{\sqrt{\frac{\beta_p}{\beta_n}}\left(V_{dd} -\left|V_{th,p}\right|\right)+V_{th,n}}{\left(1+\sqrt{\frac{\beta_p}{\beta_n}}\right)}
	\end{equation}	
	
	Zauważmy, że dla procesu o równych napięciach progowych i tranzystorów o tej samej $\beta$, wynik jest dość intuicyjny:
	\begin{equation}
	V_{out} =\frac{1}{2} V_{dd}
	\end{equation}
	
	\begin{circuitikz}
		\draw [color=black, thick]
			(0, 0) node[op amp] (opamp) {}
			(opamp.-) to[R] (-3, 0.5)
			(opamp.-) to[short,*-] ++(0,1.5) coordinate (leftR)
			to[R] (leftR -| opamp.out)
			to[short,-*] (opamp.out)
			;
	\end{circuitikz}
	\section{Wnioski z projektu wstępnego.}
	\begin{wniosek}{Zwiększanie prądu referencyjnego oznacza większy $V_{DS,sat}$, co z kolei oznacza, że wyjście wzmacniacza $A_1$ musi się obniżyć mocniej. To oznacza trudniejsze wymaganie na zakres napięć wyjściowych.}
	\end{wniosek}
	\begin{wniosek}{Zwiększanie napięcia referencyjnego, aby skompensować niskie wyjście wzmacniacza $A_1$ oznacza, że wyjście wzmacniacza $A_2$ będzie musiało z kolei pójść mocniej w górę.}
	\end{wniosek}
	\begin{wniosek}{Segmentowe połączenie pozwala nie tylko zmniejszyć minimalne L0, ale też łączną liczbę tranzystorów, potrzebnych do wykonanania zadania.}
	\end{wniosek}
	\begin{wniosek}{}
	\end{wniosek}

	\chapter{Pomiary parametrów konwertera.}
	{ Do przeprowadzenia pomiarów konwertera przygotowano modele w Verilog-AMS. 
	\begin{itemize}
		\item Idealny 8-bitowy DAC,
		\item cyfrowy generator 8-bitowych słów : sinus i pełen zakres
		\item Idealny 10-bitowy ADC,
		\item 10-bitowy subtraktor cyfrowy
	\end{itemize}}
	\section{Architektura testu.}
	{ }
		\begin{figure}[!htb]
		\centering
		\begin{tikzpicture}[auto]
		\node [block, node distance = 2cm] (fir) {Generator cyfrowy};
		\node [block, below of = fir, node distance = 2cm] (converter) {8-bit DAC};
		\node [block, below of = converter, node distance = 2 cm](sh){10-bit ADC};
		\node [block, below of = sh, node distance = 2 cm](afilter){Subtraktor};
		\node [signal, below of = afilter, node distance = 2cm] (out) {Błąd};
		
		\draw [->] (fir) -- node {} (converter);
		\draw [->] (converter) -- node {} (sh);
		\draw [->] (sh) -- node {} (afilter);
		\draw [->] (afilter) -- node {} (out);
		\end{tikzpicture}
		\caption{Schemat blokowy systemu konwersji sygnału.}
	\end{figure}


	\section{Układ testujący do wyznaczenia y.}
	\section{Układ testujący do wyznaczenia z.}
	
	\chapter{Projekt fizyczny.}
	\section{Hierarchizacja projektu.}
	\section{Widok topografii.}
	\section{Widok topografii układu 2.}
	\section{Widok topografii układu 3.}
	\section{Widok topografii układu 4.}
	
	\chapter{Symulacje i pomiary po ekstrakcji.}
	
	\chapter{Podsumowanie.}
	\section{Rozmiar układu, liczba tranzystorów.}
	\section{Weryfikacja osiągniętych celów.}
	\section{Wnioski końcowe.}	

	
	\appendix
	%insert chapters	
.	\chapter{Modele AMS}
	\section{Idealny 8-bit DAC.}
	\section{Idealny 10-bit ADC.}
	\section{Generator cyfrowy.}
	
	\chapter{Weryfikacja doboru długości kanału na podstawie charakterystyk niedopasowania.}
	{	Producent technologii dostarcza wzory opisujące odchylenie standardowe parametrów urządzeń, które zostały zaprojektowane jako identyczne. Do generacji prądów o binarnych wagach wykorzystujemy lustra prądowe, składające się z kolejno $1, 2, 4,..., 2^{N-1}$. Błąd napięciowego sygnału wyjściowego musi być mniejszy niż $\frac{LSB}{2}$. Zakładając, że napięciowy sygnał wejściowy jest proporcjonalny do sumy tych prądów:
		\begin{equation}
		V_{out} = A \sum_{i=0}^{N-1} b_{i}I_{i}
		\end{equation}
		LSB dla konwertera wynosi, gdzie FS oznacza pełen zakres sygnału wyjściowego:
		\begin{equation}
		LSB_v = \frac{FS_Vout}{2^{N-1}}
		\end{equation}
		Dla prądów:
		\begin{equation}
		LSB_i = \frac{LSB_v}{A}
		\end{equation}
		
		W związku z tym faktem zakłada się, że błąd każdego z prądów również musi być mniejszy od $\frac{LSB}{2}$. Prąd i-tej gałęzi jest sumą $2^{i}$ prądów, przepływających przez $2^{i}$ identycznych tranzystorów. Aby błąd sumarycznego prądu każdej gałęzi był mniejszy niż LSB/2, suma błędów niedopasowania wszystkich tranzystorów w każdej gałęzi musi być mniejsza niż LSB/2. 
		\begin{equation}
		\sigma \left( \Delta I_{i,ds} \right) < \frac{LSB_i}{2}
		\end{equation}
		Łatwo zauważyć, że największy błąd jest dla ostatniej gałęzi z największą liczbą tranzystorów: $n=2^{N-1}=2^7$. Minimalny prąd referencyjny w pierwszej gałęzi ma wartość LSB.
		\begin{equation}
		n \sigma \left( \Delta I_{N-1,ds} \right) LSB_i = 2^{N-1} \sigma \left( \Delta I_{N-1,ds} \right) LSB_i < \frac{LSB_i}{2}
		\end{equation}
		
		\begin{equation}
		\sigma \left( \Delta I_{N-1,ds} \right) < 2^{-N}
		\end{equation}
		
		Z \cite{ams_match_params} mamy:
		\begin{equation}
		\sigma^2 \left( \frac{\Delta I_{ds}}{I_{ds}}\right) = 	\sigma^2 \left( \frac{\Delta \mu}{\mu}\right) + \sigma^2 \left( \frac{2\Delta V_T}{V_{gs} - V_{T}}\right)
		\end{equation}
		
		\begin{equation}
		3 \sigma \left(\Delta \mu \right) = \frac{ \sqrt{2} A_{\beta}  } { \sqrt{ \left(W-A_{\beta W}\right)\left(L-A_{\beta} L\right)}}
		\end{equation}
		
		\begin{equation}
		3 \sigma \left(\Delta V_T \right) = \frac{ \sqrt{2} A_{V_T}  } { \sqrt{ \left(W-A_{VTW}\right)\left(L-A_{VTL}\right)}}
		\end{equation}
		Wykreślmy błąd prądu przy pomocy Octave, znajdziemy minimalne L.
		
	}
	
	
	
	
	
	\begin{thebibliography}{9}
		%%moje z knovela
		\bibitem{integconv} 
		Jespers, Paul G.A.. (2001). 
		\textit{Integrated Converters - D to A and A to D Architectures, Analysis and Simulation.}
		Oxford University Press
		
		\bibitem{cmosanal} 
		Allen, Phillip E. Holberg, Douglas R.. (2012)  
		\textit{CMOS Analog Circuit Design (3rd Edition). }
		Oxford University Press.
		
		\bibitem{vlsidesign} 
		Das, Debaprasad. (2015).
		\textit{VLSI Design (2nd edition)}
		Oxford University Press.
		%%z opisu tematu od profa allen cmos sie pokrywa z moim wyborem
		\bibitem{plassche} 
		R. Plassche (2001). 
		\textit{Scalone przetworniki analogowo-cyfrowe i cyfrowo-analogowe,}
		WKŁ 2001
		
		\bibitem{bdgp_1}
		Design of an improved Bandgap Reference in
		180nm CMOS Process Technology

		\bibitem{bdgp_2}
		Design and implementation of curvature corrected
		bandgap voltage reference - 1.1V using 180nm Technology

		\bibitem{ams_proc_params}
		$0.18\mu m$ HV CMOS Process Parameters, Rev. 3.0, AMS AG

		\bibitem{ams_match_params}
		$0.18\mu m$ HV CMOS Matching Parameters, Rev. 1.0, AMS AG
		
	\end{thebibliography}

\end{document}
