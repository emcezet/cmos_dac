\documentclass[10pt,a4paper]{report}
 \usepackage{booktabs}
\usepackage{float}
\usepackage{tabularx}
\usepackage{amsmath}
\usepackage{amsfonts}
\usepackage{amssymb}
\usepackage{graphicx}


\usepackage{polski}
\usepackage[polish]{babel}
\usepackage[utf8]{inputenc}
\usepackage[polish]{babel}

\usepackage{hyperref}
\hypersetup{
	colorlinks=true, %set true if you want colored links
	linktoc=all,     %set to all if you want both sections and subsections linked
	linkcolor=red,  %choose some color if you want links to stand out
}

\usepackage{tikz}
\def\checkmark{\tikz\fill[scale=0.4](0,.35) -- (.25,0) -- (1,.7) -- (.25,.15) -- cycle;} 
\usepackage{color, colortbl}



\usepackage{geometry}
\geometry{legalpaper, margin=1in}

\definecolor{Gray}{gray}{0.9}
\setcounter{secnumdepth}{4}

\title{Notatki do pracy inżynierskiej. \\ 8-bitowy przetwornik cyfrowo-analogowy.}
\date{2018}
\author{Michal Czyz \\ student, WETI PG\\}

\begin{document}
	\maketitle
	
	\tableofcontents
	\newpage
	\chapter{Preliminaria.}
	\section{Abstract.}
	\section{Abstrakt.}
	\section{Podziękowania.}
	
	\chapter{Przetworniki cyfrowo-analogowe.}
	\section{Wstęp.}
	\subsection{Zasada działania.}
	\subsection{Podział konwerterów.}
	\subsubsection{Szeregowe.}
	\subsubsection{Równoległe.}
	\subsubsection{Nyquista.}
	\subsubsection{Nadpróbkujące.}
	
	\section{Przestrzeń projektowa.}
	\subsection{title}
	\subsection{Miary jakości.}
	\subsubsection{Liniowość.}
	\subsubsection{Rozdzielczość.}
	\subsubsection{Monotoniczność.}
	\subsubsection{Inne miary.}
	
	\chapter{Schemat elektryczny.}
	\section{Schemat/opis.}
	\subsection{Dobór kryteriów prowadzących do tego rozwiązania.}
	\subsection{Schemat.}
	\section{Dowód poprawności działania.}

	\chapter{Pomiary parametrów.}
	\section{środowisko.}
	\section{Układ testujący do wyznaczenia x.}
	\section{Układ testujący do wyznaczenia y.}
	\section{Układ testujący do wyznaczenia z.}
	
	\chapter{Realizacja topografii.}
	\section{środowisko.}
	\section{Technologia.}
	\section{Hierarchizacja projektu.}
	\subsection{Moduł 1.}
	\subsection{Moduł 2.}
	\subsection{Moduł 3.}
	\subsection{Moduł 4.}
	
	\section{Pomiary parametrów po ekstrakcji.}
	\subsection{Sposób pomiaru}
	Jak przed ekstrakcją.
	\subsection{Rozmiar układu, liczba tranzystorów.}

	\chapter{Podsumowanie}
	\section{Weryfikacja osiągniętych celów.}
	
	\chapter{Dodatek A.}
	\section{Niepowiązane z pracą.}
	Może jakieś wnioski lub uwagi najdą w trakcie prac.
	
	\begin{thebibliography}{9}
		%%moje z knovela
		\bibitem{integconv} 
		Jespers, Paul G.A.. (2001). 
		\textit{Integrated Converters - D to A and A to D Architectures, Analysis and Simulation.}
		Oxford University Press
		
		\bibitem{cmosanal} 
		Allen, Phillip E. Holberg, Douglas R.. (2012)  
		\textit{CMOS Analog Circuit Design (3rd Edition). }
		Oxford University Press.
		
		\bibitem{vlsidesign} 
		Das, Debaprasad. (2015). .
		\textit{VLSI Design (2nd edition)}
		Oxford University Press.
		%%z opisu tematu od profa
		
	\end{thebibliography}

\end{document}
