\documentclass[10pt,a4paper]{report}
 \usepackage{booktabs}
\usepackage{float}
\usepackage{tabularx}
\usepackage{amsmath}
\usepackage{amsfonts}
\usepackage{amssymb}
\usepackage{graphicx}


\usepackage{polski}
\usepackage[polish]{babel}
\usepackage[utf8]{inputenc}
\usepackage[polish]{babel}

\usepackage{indentfirst}

\usepackage{hyperref}
\hypersetup{
	colorlinks=true, %set true if you want colored links
	linktoc=all,     %set to all if you want both sections and subsections linked
	linkcolor=red,  %choose some color if you want links to stand out
}

\usepackage{tikz}
\def\checkmark{\tikz\fill[scale=0.4](0,.35) -- (.25,0) -- (1,.7) -- (.25,.15) -- cycle;} 
\usepackage{color, colortbl}

\usepackage{geometry}
\geometry{legalpaper, margin=1in}

\newcommand{\img}[4]{
	\begin{figure}[H]
		\begin{center}
			\includegraphics[width=#1 cm, keepaspectratio=true]{#2}
			\caption{#3}
			\label{#4}
		\end{center}
	\end{figure}
}

\definecolor{Gray}{gray}{0.9}
\setcounter{secnumdepth}{4}

\title{Notatki do pracy inżynierskiej. \\ 8-bitowy przetwornik cyfrowo analogowy w technologii CMOS. \\ STAN : alpha}
\date{2018}
\author{Michał Czyż \\ student, WETI PG}

\begin{document}
	\maketitle
	
	\tableofcontents
	\newpage
	\chapter{Preliminaria.}
	\section{Abstract.}
	{	Goal of this thesis... \textcolor{red}{NA KONIEC UZUPEŁNIĆ.} }

	\section{Abstrakt.}
	{	Celem niniejszej pracy jest zaprojektowanie schematu i topografii masek 8-bitowego przetwornika cyfrowo analogowego w technologii CMOS AMS 180nm. \textcolor{red}{NA KONIEC UZUPEŁNIĆ.} }
	
	\section{Podziękowania.}
	{	Dziękujemy wszystkim ... . \textcolor{red}{NA KONIEC UZUPEŁNIĆ} }
	
	\chapter{Wstęp.}
	{	Konwertery sygnałów z postaci cyfrowej na analogową (i odwrotnie) są niezbędną częścią systemów elektronicznych, ponieważ umożliwiają komunikację pomiędzy zewnętrznym, analogowym światem i cyfrowymi rdzeniami układów krzemowych \cite{integconv}. Do przykładowych zastosowań konwerterów należą m.in. generowanie sygnału wizyjnego, fonicznego lub sygnałów sterowania np. dla układów radarowych czy konwerterów mocy. }
	
	\section{Architektury konwerterów C/A. \cite{plassche}}
	{	\textbf{Przetwornik z modulacją szerokości impulsów} dokonuje porównania słowa cyfrowego z liniowo rosnącym cyfrowym słowem odniesienia. Wygenerowany w ten sposób impuls o szerokości zależnej od wartości przetwarzanego słowa poddawany jest filtracji dolnoprzepustowej. Do wad tego rozwiązania należą mała szybkośc przetwarzania i konieczność stosowania filtów o wysokim tłumieniu w paśmie zaporowym. }
	
	{	\textbf{Przetwornik integracyjny z całkowaniem liniowym} również wykorzystuje mechanizm porównywania liczb cyfrowych do wygenerowania sygnału o modulowanej szerkości, który jest poddawany scałkowaniu, a następnie spróbkowaniu przez układ próbkująco pamiętający. Celem zwiększenia szybkości działania układu dokonuje się podzielenia słowa bitowego na część starszą i młodszą, a przetwarzanie obu części odbywa się równolegle. }
	
	{	\textbf{Przetworniki sieciowe ze skalowaniem} dokonują zamiany słowa cyfrowego na napięcia, prądy lub ładunki proporcjonalne (ważone) do wartości tego słowa. Sygnały z poszczególnych gałęzi sieci są sumowane, a sygnał wyjściowy podlega konwersji i/lub kondycjonowaniu do zadanej formy. Udaje się uzyskać rozdzielczość do 10-bitów.}
	
	{	\textbf{Przetwornik z kształtowaniem szumu} wykorzystuje filtry nadpróbkujące, układy kształtowania szumu, 1-bitowy przetwornik C/A i analogowy filtr wyjściowy. Uzyskuje się wysoką precyzję 16-18 bitów. \textcolor{red}{Czy przetworniki Nyquista, $\Sigma \Delta$ i nadpróbkujące to dokładnie to samo, bo tak to wygląda?} }
	
	\section{Klasyfikacje.}
	{	Dokonuje się kilku podziałów konwerterów. Ze względu na liczbę przetwarzanych bitów wyróżnia się: \textbf{szeregowe}, czyli takie, które dokonują konwersji słowa cyfrowego bit po bicie oraz \textbf{równoległe}, czyli takie, które dokonują konwersji całego słowa jednocześnie. Jeżeli sygnał wyjściowy przetwornika jest stały w czasie dla ustalonego i podtrzymywanego słowa cyfrowego, to nazywamy taki przetwornik \textbf{statycznym}, w przeciwieństwie do przetworników \textbf{dynamicznych}, których sygnał wyjściowy zanika i wymaga odświeżania. }

	\section{Parametry przetworników C/A.}
	{	Do podstawowych parametrów przetwornika należą (jest tego dużo więcej):
		\begin{itemize}
			\item Dokładność bezwzględna.
			\item Dokładność względna.
			\item Nieliniowość różniczkowa (DNL).
			\item Nieliniowość całkowa (INL). 
			\item Rozdzielczość.
			\item Przesunięcie zera.
			\item Współczynnik wrażliwości temperaturowej.
			\item Stosunek sygnału do szumu.
			\item Monotoniczność.
			\item Maksymalna częstotliwośc próbkowania,
			\item PSRR.
		\end{itemize} 
	}

	{	Kluczowym dla poprawnego działania konwertera skalowanego jest stabilne źródło napięcia odniesienia. Układy wytwarzane w technologii CMOS są wrażliwe na rorzut technologiczny parametrów elementów elektronicznych, zmieniają się ich właściwości pod wpływem zmian temperatury lub zmian wartości napięcia zasilania. Aby zapewnić stabilność napięcia odniesienia stosuje się układ bandgap. }

	\chapter{Zaprojektowany konwerter.}
	\section{Wymagania projektowe.}
	{	Narzucone wymagania na przetwornik:
		\begin{itemize}
			\item zasilanie 1.8V,
			\item CMOS AMS 180nm,
			\item szybkość konwersji powyżej 4MS,
			\item 8-bitowy.
		\end{itemize} }
	
	{	Dodatkowe wymagania:
		\begin{itemize}
			\item Sygnał wyjściowy to napięcie o zakresie co najmniej 0-1V.
			\item Dopuszczalne obciążenie: rezystancyjne co najmniej $50\Omega$.
		\end{itemize}
	}
	\section{Wybór architektury.}
	{	Do zrealizowania wybrano konwerter skalujący prąd. Ogólny schemat takiego konwertera:
		\img{20}{../visio/currentscale.pdf}{Schemat konwertera skalującego prąd}{currentscale}
	}

	{	Podstawowe i niezbędne bloki konwertera to:
		\begin{itemize}
			\item Blok konwersji.
			\item Rejestr wejściowy.
			\item Źródło napięcia odniesienia typu Bandgap.
		\end{itemize}
		\img{20}{../visio/blokschkonca.pdf}{Blokowy schemat konwertera C/A.}{blokschkonca}
		Dodatkowymi blokami mogą być:
		\begin{itemize}
			\item Cyfrowy interfejs, np. SPI, I2C.
			\item Blok kontroli (wysyłanie komend przez interfejs)
			\item Programowalne wzmocnienie wyjścia.
			\item Blok autokalibracji.
		\end{itemize}	
	}

	\chapter{Schemat elektryczny.}
	\section{Układ bandgap.}
	\section{Sięć ważonych prądów.}
	\section{Wyjściowy wzmacniacz transkonduktancyjny.}

	\section{Dowód poprawności działania.}

	\chapter{Pomiary parametrów.}
	\section{środowisko.}
	\section{Układ testujący do wyznaczenia x.}
	\section{Układ testujący do wyznaczenia y.}
	\section{Układ testujący do wyznaczenia z.}
	
	\chapter{Implementacja.}
	\section{środowisko.}
	\section{Technologia.}
	\section{Hierarchizacja projektu.}
	\subsection{Moduł 1.}
	\subsection{Moduł 2.}
	\subsection{Moduł 3.}
	\subsection{Moduł 4.}
	
	\section{Pomiary parametrów po ekstrakcji.}
	\subsection{Sposób pomiaru}
	Jak przed ekstrakcją.
	\subsection{Rozmiar układu, liczba tranzystorów.}

	\chapter{Podsumowanie}
	\section{Weryfikacja osiągniętych celów.}
	
	
	\appendix
	%insert chapters	
.
	\begin{thebibliography}{9}
		%%moje z knovela
		\bibitem{integconv} 
		Jespers, Paul G.A.. (2001). 
		\textit{Integrated Converters - D to A and A to D Architectures, Analysis and Simulation.}
		Oxford University Press
		
		\bibitem{cmosanal} 
		Allen, Phillip E. Holberg, Douglas R.. (2012)  
		\textit{CMOS Analog Circuit Design (3rd Edition). }
		Oxford University Press.
		
		\bibitem{vlsidesign} 
		Das, Debaprasad. (2015). .
		\textit{VLSI Design (2nd edition)}
		Oxford University Press.
		%%z opisu tematu od profa allen cmos sie pokrywa z moim wyborem
		\bibitem{plassche} 
		R. Plassche (2001). 
		\textit{Scalone przetworniki analogowo-cyfrowe i cyfrowo-analogowe,}
		WKŁ 2001
	\end{thebibliography}

\end{document}
