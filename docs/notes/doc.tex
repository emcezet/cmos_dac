\documentclass[10pt,a4paper]{report}
 \usepackage{booktabs}
\usepackage{float}
\usepackage{tabularx}
\usepackage{amsmath}
\usepackage{amsfonts}
\usepackage{amssymb}
\usepackage{graphicx}
\usepackage{amsthm}

\usepackage{polski}
\usepackage[polish]{babel}
\usepackage[utf8]{inputenc}
\usepackage[polish]{babel}

\usepackage{indentfirst}
\usepackage{ifthen}
\usepackage{hyperref}
\hypersetup{
	colorlinks=true, %set true if you want colored links
	linktoc=all,     %set to all if you want both sections and subsections linked
	linkcolor=red,  %choose some color if you want links to stand out
}

\usepackage{tikz}
\def\checkmark{\tikz\fill[scale=0.4](0,.35) -- (.25,0) -- (1,.7) -- (.25,.15) -- cycle;} 
\usepackage{color, colortbl}

\usepackage{pgfplots}       % <-- required in preamble
\usepackage{pgfplotstable}       % <-- required in preamble
%\usepackage{SIunits}        % <-- required in preamble
\pgfplotsset{compat=newest} % <-- optional in preamble


\usepackage[americancurrents]{circuitikz}


\usetikzlibrary{shapes,arrows,calc}

\tikzstyle{block} = [rectangle, draw, text width=3cm, 
text centered, rounded corners, 
minimum height=1.5cm, node distance=2cm]

\tikzstyle{signal} = [rectangle, text width=2.5cm, 
text centered, rounded corners, 
minimum height=1cm, node distance=2cm]

\tikzstyle{line} = [draw, -latex']

\usepackage{geometry}
\geometry{legalpaper, margin=1in}

\theoremstyle{definition}
\newtheorem{notation}{Notacja}[section]

\theoremstyle{definition}
\newtheorem{definition}{Definicja}[section]

\theoremstyle{definition}
\newtheorem{przyklad}{Przykład}[section]

\theoremstyle{definition}
\newtheorem{twierdzenie}{Twierdzenie}[section]

\theoremstyle{definition}
\newtheorem{wniosek}{Wniosek}[section]

\renewcommand{\baselinestretch}{1.5} 

\newcommand{\img}[4]{
	\begin{figure}[H]
		\begin{center}
			\includegraphics[width=#1 cm, keepaspectratio=true]{#2}
			\caption{#3}
			\label{#4}
		\end{center}
	\end{figure}
}

\definecolor{Gray}{gray}{0.9}
\setcounter{secnumdepth}{4}

\title{Notatki do pracy inżynierskiej. \\ 8-bitowy przetwornik cyfrowo analogowy w technologii CMOS. \\ STAN : alpha}
\date{2018}
\author{Michał Czyż \\ student, WETI PG}

\begin{document}
	\maketitle
	
	\tableofcontents
	\newpage
	\chapter{Preliminaria.}
	\section{Abstract.}
	{	Goal of this thesis... \textcolor{red}{NA KONIEC UZUPEŁNIĆ.} }

	\section{Abstrakt.}
	{	Przetworniki cyfrowo-analogowe są niezbędną częścią wielu systemów mikroelektronicznych. Celem niniejszej pracy jest zaprojektowanie schematu i topografii masek 8-bitowego przetwornika cyfrowo analogowego w technologii CMOS AMS 180nm. Zaprojektowany przetwornik to segmentowe połączenie dwóch 4-bitowych równoległych konwerterów ze skalowaniem prądu, zdolny do przetwarzania $4MSPS$ z całkowym błędem nieliniowości INL mniejszym niż $0.5LSB$. Wykonano serię symulacji służących do pomiaru parametrów konwertera.}
	
	\section{Podziękowania.}
	{	Dziękujemy wszystkim ... . \textcolor{red}{NA KONIEC UZUPEŁNIĆ} }
	
	\chapter{Wstęp.}
	{	W tym rozdziale dokonano zbioru najistotniejszych pojęć niezbędnych do zrozumienia sposobu działania konwerterów danych i zagadnień związanych z przetwarzaniem sygnałów. Dokonano zwięzłego opisu wybranych architektur przetworników oraz przytoczono definicje parametrów przetworników.}
	

	\section{Systemy konwersji cyfrowo analogowej.}
	{	Konwertery sygnałów z postaci cyfrowej na analogową (i odwrotnie) są niezbędną częścią systemów elektronicznych, ponieważ umożliwiają komunikację pomiędzy zewnętrznym, analogowym światem i cyfrowymi rdzeniami układów krzemowych \cite{integconv}. Do przykładowych zastosowań konwerterów danych należą m.in. generowanie sygnału wizyjnego, fonicznego lub sygnałów sterowania np. dla układów radarowych, konwerterów mocy lub miernictwa. }
	
	\subsection{Twierdzenie Nyquista.}
	
	\begin{twierdzenie}{Niech sygnał $S$ zajmuje pasmo częstotliwości $B =\left(0, f_{max}\right) $. Sygnał $S$ może być przetwarzany bezbłędnie, wtedy i tylko wtedy, gdy częstotliwość próbkowania $f_{sample}$ jest co najmniej dwukrotnie większa od górnego zakresu pasma częstotliwości $f_{max}$. }
	$$
		f_{sample} \ge 2 f_{max}
	$$
	\end{twierdzenie}
	
	\begin{przyklad}{Dany jest konwerter danych, który przetwarza 4 miliony próbek na sekundę ($4MSPS$). Zgodnie z twierdzeniem Nyquista maksymalna częstotliwość przetwarzanego sygnału wynosi:}
		\begin{equation}
			f_{max} = \frac{f_{sample}}{2} = \frac{4M}{2} = 2MHz
		\end{equation}
	\end{przyklad}
	
	\subsection{Filtracja sygnału wejściowego.}
	{	W przypadku gdy twierdzenie Nyquista nie jest spełnione w widmie sygnału wyjściowego obserwuje się imaże (zwielokrotnienia) widm, które powodują zniekształcenie sygnału. Z tego powodu do poprawnego funkcjonowania systemu konwersji cyfrowo analogowej przed blokiem konwersji należy umieścić cyfrowy filtr dolnoprzepustowy, którego zadaniem jest ograniczenie pasma sygnału wejściowego w taki sposób, aby spełnione było twierdzenie Nyquista. Tego rodzaju filtr jest nazywany filtrem antyaliasowym. }
	
	\begin{przyklad}{Dany jest konwerter danych, który przetwarza z szybkością $4MSPS$. Filtr antyaliasowy o transmitancji $H(f)$ powinien mieć charakterystykę amplitudową:}
		\[   
		|H(f)| = 
		\begin{cases}
			1 &\quad \text{dla} f \le 2MHz\\
			0 &\quad \text{dla} f>2MHz\\
		\end{cases}
		\]		
	\end{przyklad}
	\begin{figure}[!htb]
	\centering
	\begin{tikzpicture}
	\draw[->] (-4,0) -- (4,0) node[below] {$f$};
	\draw[->] (0,-1) -- (0,2) node[right] {$|H(f)|$};
	\draw[scale=1,domain=-2:2,smooth,variable=\x,blue] plot ({\x},{1});
	\draw[scale=1,domain=-4:-2,smooth,variable=\x,blue] plot ({\x},{0});
	\draw[scale=1,domain=2:4,smooth,variable=\x,blue] plot ({\x},{0});
	\draw[scale=1,domain=0:1,smooth,variable=\y,blue] plot ({-2},{\y});
	\draw[scale=1,domain=0:1,smooth,variable=\y,blue] plot ({2},{\y});
	\end{tikzpicture}
	\end{figure}
		
	\subsection{Filtracja sygnału wyjściowego.}
	{	W trakcie konwersji wyjście konwertera znajduje się w stanie przejściowym, co wprowadza
	zniekształcenia sygnału wyjściowego. Wynika to zazwyczaj z faktu, że przejściu pomiędzy różnymi słowami cyfrowymi towarzyszy przełączenie jednego lub więcej kluczy. Ochronę przed zniekształceniami sygnału i impulsami szpilkowymi osiąga się dzięki stosowaniu układu próbkująco pamiętającego i filtru wyjściowego. Ten ostatni popularnie nazywany jest filtrem rekonstrukcyjnym. }
	\begin{przyklad}{Dany jest konwerter danych, który przetwarza $4MSPS$. Filtr rekonstrukcyjny powinien być filtrem dolnoprzepustowym o częstotliwości granicznej równej $2MHz$.}
	\end{przyklad}

	\subsection{Systemy przetwarzania cyfrowo analogowego.}
	{	Ogólny schemat blokowy systemu przetwarzania sygnału z postaci cyfrowej na analogową został przedstawiony na rysunku \ref{blokca}.
		}
	\begin{figure}[!htb]
	\centering
		\begin{tikzpicture}[auto]
		\node [signal] (in) {Sygnał \\ cyfrowy};
		\node [block, below of = in, node distance = 2cm] (fir) {FDP antyaliasing};
		\node [block, below of = fir, node distance = 2cm] (converter) {Blok \\ konwersji C/A};
		\node [block, below of = converter, node distance = 2 cm](sh){Układ próbkująco-pamiętający S\&H};
		\node [block, below of = sh, node distance = 2 cm](afilter){FDP rekonstrukcja};
		\node [signal, below of = afilter, node distance = 2cm] (out) {Sygnał analogowy};
		
		\draw [->] (in) -- node {} (fir);
		\draw [->] (fir) -- node {} (converter);
		\draw [->] (converter) -- node {} (sh);
		\draw [->] (sh) -- node {} (afilter);
		\draw [->] (afilter) -- node {} (out);
		\end{tikzpicture}
	\label{blokca}
	\caption{Schemat blokowy systemu konwersji sygnału.}
	\end{figure}
	
	
	\section{Przegląd architektur konwerterów C/A. \cite{cmosanal} \cite{plassche}}
	{	Dokonuje się kilku podziałów konwerterów. Ze względu na liczbę przetwarzanych bitów wyróżnia się: \textbf{szeregowe}, czyli takie, które dokonują konwersji słowa cyfrowego bit po bicie oraz \textbf{równoległe}, czyli takie, które dokonują konwersji całego słowa jednocześnie. Jeżeli sygnał wyjściowy przetwornika jest stały w czasie dla ustalonego i podtrzymywanego słowa cyfrowego, to nazywamy taki przetwornik \textbf{statycznym}, w przeciwieństwie do przetworników \textbf{dynamicznych}, których sygnał wyjściowy zanika i wymaga odświeżania. }
	
	{	\textbf{Przetwornik z modulacją szerokości impulsów} dokonuje porównania słowa cyfrowego z liniowo rosnącym cyfrowym słowem odniesienia. Wygenerowany w ten sposób impuls o szerokości zależnej od wartości przetwarzanego słowa poddawany jest filtracji dolnoprzepustowej. Do wad tego rozwiązania należą mała szybkość przetwarzania i konieczność stosowania filtrów o wysokim tłumieniu w paśmie zaporowym. }
	
	{	\textbf{Przetwornik integracyjny z całkowaniem liniowym} również wykorzystuje mechanizm porównywania liczb cyfrowych do wygenerowania sygnału o modulowanej szerokości, który jest poddawany scałkowaniu, a następnie próbkowaniu przez układ próbkująco pamiętający. Celem zwiększenia szybkości działania układu dokonuje się podzielenia słowa bitowego na część starszą i młodszą, a przetwarzanie obu części odbywa się równolegle. }
	
	{	\textbf{Przetworniki sieciowe ze skalowaniem} dokonują zamiany słowa cyfrowego na napięcia, prądy lub ładunki proporcjonalne do wartości tego słowa. Ważone dwójkowo sygnały z poszczególnych gałęzi sieci są sumowane, a sygnał wyjściowy podlega konwersji i/lub kondycjonowaniu do zadanej formy. Udaje się uzyskać rozdzielczość do 10-bitów.}
	
	{	\textbf{Przetwornik z kształtowaniem szumu} wykorzystuje filtry nad-próbkujące, układy kształtowania szumu, 1-bitowy przetwornik C/A i analogowy filtr wyjściowy. Uzyskuje się wysoką precyzję 16-18 bitów. }
	
	{	\textbf{Przetwornik szeregowy z redystrybucją ładunku} wykorzystuje układ dwóch połączonych równolegle identycznych kondensatorów i kilka kluczy. Kondensator wejściowy jest ładowany lub rozładowywany w zależności od wartości kolejnych bitów, a następnie dołączany do kondensatora wyjściowego, co pozwala na dodanie lub odjęcie ładunku, a więc zwiększenie lub zmniejszenie wartości napięcia wyjściowego.
	}
	
	{	\textbf{Przetwornik algorytmiczny potokowy} składa się z sumatorów i bloków opóźnienia, które pozwalają na dodawanie przeskalowanych napięć referencyjnych w kolejnych blokach. \textcolor{red}{przepisać} }

	{	\textbf{Przetwornik algorytmiczny iteracyjny} w węźle układu próbkująco pamiętającego dodaje kolejne bity. \textcolor{red}{przepisać} }

	{ 	\textbf{Przetworniki segmentowe} to przetworniki składające się z połączenia dwóch lub więcej przetworników. Zazwyczaj łączone są przetworniki o małej liczbie bitów w większe, aby zwiększyć rozdzielczość. }

	{	Pierwszym krokiem w projekcie przetwornika jest wybór architektury na podstawie analizy obecnego stanu wiedzy technicznej oraz wymagań dotyczących rozdzielczości, szybkości, powierzchni i pobieranej mocy. Po zapoznaniu się z literaturą wybór padł na konwerter skalujący prąd. Najistotniejsze motywacje tego wyboru to prostota schematu, wysoka szybkość działania \cite{cmosanal} oraz fakt, że poprzednie prace uzyskały do 10 bitów rozdzielczości \cite{plassche}.	Autorzy \cite{cmosanal} jako główne wady podają duży rozmiar oraz duży rozstęp wielkości elementów.}
	\section{Równoległy przetwornik skalujący prąd.}
	
	{	Blokowy schemat konwertera skalującego prąd znajduje się na rysunku \ref{currentscale}. Zasada działania jest następująca: słowo cyfrowe jest sygnałem sterującym sieć, która może wytwarzać prądy o różnych wartościach. Typowo, przypisuje się prądom wagi dwójkowe, a kolejne bity słowa cyfrowego decydują o włączeniu/wyłączeniu gałęzi prądowej, co oznacza, że po zsumowaniu prądów wyjściowych otrzymujemy prąd o wartości  bezwzględnej odpowiadającej wartości zakodowanej w słowie cyfrowym. W zależności od wymagań na sygnał wyjściowy, dokonuje się konwersji prądu wyjściowego na napięcie. }


\begin{figure}[!htb]
	\centering
	\begin{tikzpicture}[auto]
	\node [signal, node distance = 2cm] (digint) {N-bitowe słowo cyfrowe};
	\node [block, below of = digint, node distance = 2cm] (network) {Sieć skalująca prąd};
	\node [block, left of = network, node distance = 4cm] (iref1) {Źródło prądu referencyjnego};
	\node [block, below of = network, node distance = 2cm] (conv) {Konwerter prąd/napięcie};
	\node [block, left of = conv, node distance = 4cm] (vref1) {Źródło napięcia referencyjnego};
	\node [signal, below of = conv, node distance = 2cm] (out) {Sygnał \\ analogowy};
	
	\draw [->] (iref1) -- node {$I_{ref}$} (network);
	\draw [->] (digint) -- node {$B=\{b_0, b_1, \dots b_{N-1} \}$} (network);
	\draw [->] (vref1) -- node {$V_{ref}$} (conv);
	\draw [->] (network) -- node {$I_{out}=f(I_{ref},B)$} (conv);
	\draw [->] (conv) -- node {$V_{out}=g(V_{ref},I_{out})$} (out);
	\end{tikzpicture}
	\caption{Schemat blokowy konwertera skalującego prąd.}
\end{figure}

	\section{Segmentowy równoległy przetwornik skalujący prąd.}
{	Prąd wyjściowy przetwornika skalującego prąd ze skalowaniem ważonym dwójkowo wynosi:
	\begin{equation}
	I_{out} = \sum_{i=0}^{N-1} b_i I_{i} = I_{ref} \sum_{i=0}^{N-1} b_i 2^{i}
	\end{equation}
	Zauważmy, że sumując wyjścia dwóch przetworników otrzymujemy w ogolności
	\begin{equation}
	I_{out,MN} = I_{out,M} + I_{out,N}
	= I_{ref,M} \sum_{i=0}^{M-1} b_i 2^{i} + I_{ref,N} \sum_{i=0}^{N-1} b_i 2^{i}
	\end{equation}
	W szczególności, gdy zachodzi $I_{ref,N} = 2^M I_{ref} $:
	\begin{equation}
	I_{out,MN} = I_{ref,M}\left(\sum_{i=0}^{M-1} b_i 2^{i} + 2^M \sum_{i=0}^{N-1} b_i 2^{i}\right) 
	\end{equation}
	\begin{equation}
	I_{out,MN} = I_{ref,M}\left(\sum_{i=0}^{M-1} b_i 2^{i} + \sum_{i=M}^{M+N-1} b_i 2^{i}\right) 
	\end{equation}
	\begin{equation}
	I_{out,MN} = I_{ref,M} \sum_{i=0}^{M+N-1} b_i 2^{i}
	\end{equation}
	Jest to równoważne wyrażeniu na prąd wyjściowy konwertera o $M+N$ bitach rozdzielczości.
	\begin{przyklad}{Niech dany będzie M=4 bitowy konwerter o prądzie referencyjnym $I_{ref,M}$. Sumując wyjście tego przetwornika z przetwornikiem o N=4 bitach i prądzie referencyjnym $ I_{ref,N} = 2^4 \cdot I_{ref,M} $, tworzy się przetwornik o $M+N=8$ bitach.}
	\end{przyklad}	
}



	\section{Parametry przetworników C/A.}
	{	Sygnałem wejściowym dla N bitowego przetwornika cyfrowo analogowego jest słowo cyfrowe $\mathcal{B}=\{b_{N-1},b_{N-2},...,b_0\}$. Bit $b_{N-1}$ nosi miano najstarszego bitu (MSB - Most Siginificant Bit), a bit $b_0$ najmłodszego (LSB - Least Siginificant Bit). Sygnał wyjściowy to analogowe napięcie lub prąd $\mathcal{S}$ przeskalowany przez sygnał referencyjny $V_{ref}$. Zależność pomiędzy sygnałem wyjściowym i wejściowym to wtedy:
	\begin{equation}
	\mathcal{S} = V_{ref}\mathcal{B} = V_{ref} {\sum_{k=0}^{N-1} b_k2^k}
	\end{equation}	}
	
	\begin{notation}{Sygnał wyjściowy jest z zakresu:}
		$$
			S_{out} \in \langle S_{out_{min}} ;  S_{out_{max}} \rangle
		$$
	\end{notation}

	\begin{definition}{Zakres sygnału wyjściowego FS to zakres wartości, które może przyjmować sygnał wyjściowy przetwornika.}
		\begin{equation}
			FS = S_{out_{max}} - S_{out_{min}}
		\end{equation}
	\end{definition}

	\begin{definition}{Rozdzielczość R określa minimalną zmianę sygnału wyjściowego dla kolejnych słów cyfrowych. Dla idealnego przetwornika jest to wartość stała, równa stosunkowi pełnego zakresu napięcia wyjściowego do liczby poziomów (różnych słów cyfrowych). }
		\begin{equation}
			R = \frac{FS}{2^{N}-1}		
		\end{equation}
	\end{definition}
		
	\begin{definition}{Dokładność względna $\delta$ to odchylenie wartości sygnału wyjściowego w stanie ustalonym od teoretycznej prostej wyznaczonej przez pełny zakres przetwarzania. Dokładność względną nazywa się nieliniowością całkową INL.}	
	\end{definition}
		
	\begin{definition}{Nieliniowość różniczkowa DNL to zmiana sygnału wyjściowego przy przejściu o jedno słowo cyfrowe obliczone dla każdego przejścia osobno. }
	\end{definition}	
	
	\begin{definition}{Przetwornik jest monotoniczny, jeżeli sygnał wyjściowy przetwornika może być opisany przy pomocy funkcji monotonicznej.}
	\end{definition}

	\chapter{Technologia CMOS AMS 180nm.}
	\section{Czynniki wpływające na pogorszenie pracy układu.}
	{	Poprawnie zaprojektowane układy scalone charakteryzują się niewrażliwością na rozrzut technologiczny parametrów elementów elektronicznych, zmian temperatury oraz odchyłki napięcia zasilania od wartości nominalnej. Analiza zaprojektowanego konwertera pod kątem wrażliwości na te czynniki wybiega poza zakres tej pracy. }

	\section{Model kwadratowy tranzystora LV-FET z kanałem indukowanym do obliczeń ręcznych.}
	{ 	W pracy wykorzystywane są niskonapięciowe tranzystory polowe z kanałami indukowanymi typu n i p. Do obliczeń ręcznych wykorzystano uproszczony model kwadratowy. Opis tego modelu jest szeroko omówiony w \cite{cmosanal}. Dla przejrzystości pracy przytoczono niektóre oznaczenia oraz stałe technologiczne. }
	{	W obszarze nasycenia tranzystora NFET prąd drenu jest równy:
		\begin{equation}
			I_{DS} = \frac{1}{2} \mu_n C_{ox} \frac{W}{L} \left( V_{GS} - V_{TH,n} \right)^2
		\end{equation}	
		W obszarze nasycenia tranzystora PFET prąd drenu jest równy:
	
		\begin{equation}
		I_{SD} = \frac{1}{2} \mu_p C_{ox} \frac{W}{L} \left( V_{SG} - \left|V_{TH,p}\right| \right)^2
		\end{equation}	
	}

	{	Ponadto wprowadzono oznaczenia:
	
		$$
			K_i = \mu_i C_{ox}
		$$
	
		$$
			\beta_i = \mu_i C_{ox} \frac{W}{L} = K_i \frac{W}{L}
		$$
	}

	{	W dokumentacji można odnaleźć następujące wartości stałych procesu:
		$$
			V_{TH,n} = 0.355V
		$$
		
		$$
			K_n = 274 \frac{\mu A}{V^2}
		$$

		$$
			V_{TH,p} = -0.405V
		$$
	
		$$
			K_p = 56 \frac{\mu A}{V^2}
		$$	
	}

	\chapter{Projekt wstępny.}

	\section{Wymagania.}
	{	Projekt układu rozpoczyna się od sformułowania wymagań (specyfikacji) urządzenia lub systemu. Sformułowano następujące wymagania na przetwornik:
		\begin{itemize}
			\item zasilany ze źródła napięcia stałego o wartości 1.8V,
			\item zaprojektowany w technologii CMOS AMS 180nm,
			\item osiąga szybkość konwersji 4 milionów próbek na sekundę (MSPS),
			\item ma 8-bitów rozdzielczości.
	\end{itemize} }

	\section{Zaproponowana architektura.}
	{	Zaproponowano przetwornik równoległy ze skalowaniem prądu oparty o sieć luster prądowych, złożony z dwóch kaskadowo połączonych przetworników 4-bitowych. Prąd referencyjny $I_{ref}$ przepływający przez tranzystor $M_0$ jest replikowany przez tranzystory $M_1, M_2, \dots M_N$ ze współczynnikami binarnymi $1,2,\dots, 2^{N-1}$. Wzmacniacze $A_1$ i $A_2$ służą stabilizacji napięć dren źródło, niezbędne dla prawidłowej pracy luster prądowych. Tranzystory $K_1, K_2, \dots K_N$ służą za klucze. Prądy gałęzi są sumowane w węźle $A_2.-$ i konwertowane na napięcie na rezystorze $R$. W dalszej części pracy przeanalizowano 3 istotne fragmenty tego układu: lustra prądowe, klucze i wzmacniacze operacyjne. Wyprowadzono podstawowe zależności, które decydują o wymaganiach dotyczących wymienionych elementów.}

	\begin{figure}[!htb]
	\centering
	\begin{circuitikz}[scale = 0.6]
%		\draw [gray, color=gray] (0,-1) grid (20,10);
		\draw [color=black, thick]
		% Devices
		(2, 2) node[op amp, rotate = 270] (opamp) {$A_1$}
		(18, 7) node[op amp] (opamp2) {$A_2$}
		(4, 1) node[nmos, rotate =180, label={ [centered,xshift=-10, yshift = 10] {$M_0$} } ](m0){}
		(m0.D) to [short] (4,-0.5);

		\foreach \x in {1,...,4}
		{
			\pgfmathsetmacro{\offsetx}{int(3*\x+5)};
			\draw [color=black, thick]
			(\offsetx,1) node[nmos, label={ [centered,xshift=10, yshift = 10] {$M_\x$} } ](m\x){}
			(\offsetx,3) node[nmos, label={ [centered,xshift=10, yshift = 10] {$K_\x$} } ](k\x){}
			(k\x.D) to [short] (\offsetx,5)
			(m\x.S) to [short] (\offsetx,-0.5)
			;
		}
		\draw[color=black, thick]
		% Ports
		(opamp.+) to [short, -o](1.2,5)
		(opamp.-) to [short](2.7,4)
		(opamp.out) to [short](2,-0.5)
		(2,-0.5) to [short](17,-0.5)
		(5,1) to [short](16,1)
		(2.7,4) to [short](4,4)
		(4,4) to [short](4,2)
		(1.5,6) node[]{\large{\textbf{$V_{ref}$}}}
		(8,5) to [short](17,5)
		(12,5) to [short](12,7.7)
		(12,7.7) to[short](opamp2.-)
		(15,6.2) to[short,o-](opamp2.+)
		(14,6.2) node[]{\large{\textbf{$V_{ref}$}}}
		(14,7.7) to [short] (14,10)
		(14,10) to[R=$R$] (20,10)
		(20,10) to[short](20,7)
		(20,7) to[short](opamp2.out)
		(20,7) to[short,-o](21,7)
		(21,8) node[]{\large{\textbf{$V_{out}$}}}
		(4,9) to [I=${I_{ref}}$] (4,4)
		(4,8) node[ground, rotate = 180, label={ [centered,xshift=20, yshift = 5] {$GND$}}](4,10){}
		;
		
	\end{circuitikz}
	\label{4bitca}
	\caption{4-bitowy przetwornik C/A.}
\end{figure}

	\subsection{Analiza lustra prądowego z kluczem.}
	{ Projektując lustro prądowe, należy zadbać, aby tranzystory były w obszarze aktywnym, czyli:
	\begin{equation} \label{eq_cutoff}
	V_{GS} > V_{TH,n}
	\end{equation}
	oraz
	\begin{equation} \label{eq_saturation}
	V_{DS} \ge V_{GS} - V_{TH,n}
	\end{equation}
	Wtedy prąd drenu jest równy:
	\begin{equation}
	I_{DS} = \beta \left( V_{GS} - V_{TH,n} \right)^2 \left(1+\lambda V_{DS} \right)
	\end{equation}
	Zakładając, że wzmacniacz ustali $V_{GS}$ obu tranzystorów w lustrze na ten sam poziom, stosunek prądów obu tranzystorów jest równy:
	\begin{equation}
	\frac{I_o}{I_{ref}} = \frac{\left(1+\lambda V_{DS,1} \right)}{\left(1+\lambda V_{DS,0} \right)}
	\end{equation}
	Na błąd replikacji prądu duży wpływ ma odchyłka napięcia $V_{DS,1}$ od napięcia $V_{DS,0}$. Dla konfiguracji z włączonym kluczem jak na rysunku zawsze wystąpi niezerowy spadek napięcia na tranzystorze kluczującym, co jest jednym ze źródeł błędu przetwarzania. Celem zmniejszenia spadku napięcia na kluczu należy wybrać stosunkowo duże $\frac{W}{L}$.
	}

	\section{Badanie luster prądowych.}
	{	W pierwszej kolejności wykonano symulacje 4 luster prądowych stabilizowanych idealnymi wzmacniaczami operacyjnymi bez kluczy, celem wyznaczenia minimalnych rozmiarów tranzystorów  $M_1, M_2, M_3, M_4$. Konwerter będzie działał poprawnie, gdy błąd replikacji prądów będzie mniejszy niż połowa LSB. Błąd obliczano:}
	\begin{equation}
		\delta I = \frac{I_{out,measure} - I_{out,ideal}}{I_{ref}} = \frac{I_{out,measure} - 15 I_{ref}}{I_{ref}}
	\end{equation}
	\subsection{Wpływ prądu referencyjnego na błąd replikacji prądu.}
	{ Wykonano symulację DC dla następujących parametrów:
	\begin{itemize}
		\item $A_d = 100M$, $V_{off} = 0V$
		\item $I_{ref} = 100nA$, $V_{ref} = 900mV$
		\item $L_0 = 1.8\mu$, $W_0 = 3.6\mu$
	\end{itemize} }
	
	\begin{figure}[!htb]
	\centering
	\begin{tikzpicture}
	\begin{axis}[
	width=0.6\linewidth, % Scale the plot to \linewidth
	grid=major, % Display a grid
	grid style={dashed,gray!30}, % Set the style
	xlabel=$ I_{ref} $, % Set the labels
	ylabel=$ \delta I$ ,
	legend style={at={(0.5,-0.2)},anchor=north}, % Put the legend below the plot
	x tick label style={anchor=north} % Display labels sideways
	]
	\addplot[only marks, mark=star, blue] table [x={I}, y={error}, col sep=comma]{\detokenize{1_sweep_iref.csv}};
	\end{axis}
	\end{tikzpicture}
	\caption{Błąd replikacji prądu wyrażony w LSB.}
	\end{figure}
	
	\begin{wniosek}{ Błąd replikacji jest odwrotnie proporcjonalny do prądu referencyjnego.}
	\end{wniosek}

	\subsection{Wpływ napięcia referencyjnego na błąd replikacji prądu.}
		{ Wykonano symulację DC dla następujących parametrów:
		\begin{itemize}
		\item $A_d = 100M$, $V_{off} = 0V$
		\item $I_{ref} = 100nA$, $V_{ref} = 900mV$
		\item $L_0 = 1.8\mu$, $W_0 = 3.6\mu$
	\end{itemize} }
	
	\begin{figure}[!htb]
		\centering
		\begin{tikzpicture}
		\begin{axis}[
		width=0.6\linewidth, % Scale the plot to \linewidth
		grid=major, % Display a grid
		grid style={dashed,gray!30}, % Set the style
		xlabel=$ V_{ref} $, % Set the labels
		ylabel=$ \delta I$ ,
		legend style={at={(0.5,-0.2)},anchor=north}, % Put the legend below the plot
		x tick label style={anchor=north} % Display labels sideways
		]
		\addplot[only marks, mark=star, blue] table [x={I}, y={error}, col sep=comma]{\detokenize{2_sweep_vref.csv}};
		\end{axis}
		\end{tikzpicture}
		\caption{Błąd replikacji prądu wyrażony w LSB.}
	\end{figure}
	
	\begin{wniosek}{ Błąd replikacji jest słabo zależny od napięcia referencyjnego.}
	\end{wniosek}
	
	\subsection{Wpływ szerokości tranzystorów na błąd replikacji prądu.}
	{ Wykonano symulację DC dla następujących parametrów:
		\begin{itemize}
			\item $A_d = 100M$, $V_{off} = 0V$
			\item $I_{ref} = 100nA$, $V_{ref} = 900mV$
			\item $L_0 = 1\mu$, $W_0 = 1\mu$
	\end{itemize} }
		\begin{figure}[!htb]
		\centering
		\begin{tikzpicture}
		\begin{axis}[
		width=0.6\linewidth, % Scale the plot to \linewidth
		grid=major, % Display a grid
		grid style={dashed,gray!30}, % Set the style
		xlabel=$ I_{ref} $, % Set the labels
		ylabel=$ \delta I$,
		legend style={at={(0.5,-0.2)},anchor=north}, % Put the legend below the plot
		x tick label style={anchor=north} % Display labels sideways
		]
		\addplot[only marks, mark=star, blue] table [x={error (W0=1e-06) X}, y={error (W0=1e-06) Y}, col sep=comma]{\detokenize{3_step_W.csv}};
		\addplot[only marks, mark=star, red] table [x={error (W0=1.066667e-05) X}, y={error (W0=1.066667e-05) Y}, col sep=comma]{\detokenize{3_step_W.csv}};
		\addplot[only marks, mark=star, black] table [x={error (W0=2.033333e-05) X}, y={error (W0=2.033333e-05) Y}, col sep=comma]{\detokenize{3_step_W.csv}};
		\addplot[only marks, mark=star, green] table [x={error (W0=3e-05) X}, y={error (W0=3e-05) Y}, col sep=comma]{\detokenize{3_step_W.csv}};
		\end{axis}
		\end{tikzpicture}
		\caption{Błąd replikacji prądu wyrażony w LSB dla różnych szerokości.}
	\end{figure}

	
	\begin{wniosek}{ Błąd replikacji jest mniejszy dla tranzystorów o mniejszej szerokości.}
	\end{wniosek}
	
	\subsection{Wpływ długości tranzystorów na błąd replikacji prądu.}
	{ Wykonano symulację DC dla następujących parametrów:
		\begin{itemize}
			\item $A_d = 100M$, $V_{off} = 0V$
			\item $I_{ref} = 100nA$, $V_{ref} = 900mV$
			\item $L_0 = 1\mu$, $W_0 = 1\mu$
	\end{itemize} }
	
	
	\begin{figure}[!htb]
	\centering
	\begin{tikzpicture}
	\begin{axis}[
	width=0.6\linewidth, % Scale the plot to \linewidth
	grid=major, % Display a grid
	grid style={dashed,gray!30}, % Set the style
	xlabel=$ I_{ref} $, % Set the labels
	ylabel=$ \delta I$,
	legend style={at={(0.5,-0.2)},anchor=north}, % Put the legend below the plot
	x tick label style={anchor=north} % Display labels sideways
	]
	\addplot[only marks, mark=star, blue] table [x={error (L0=1e-06) X}, y={error (L0=1e-06) Y}, col sep=comma]{\detokenize{4_step_L.csv}};
	\addplot[only marks, mark=star, red] table [x={error (L0=5.666667e-06) X}, y={error (L0=5.666667e-06) Y}, col sep=comma]{\detokenize{4_step_L.csv}};
	\addplot[only marks, mark=star, black] table [x={error (L0=1.033333e-05) X}, y={error (L0=1.033333e-05) Y}, col sep=comma]{\detokenize{4_step_L.csv}};
	\addplot[only marks, mark=star, green] table [x={error (L0=1.5e-05) X}, y={error (L0=1.5e-05) Y}, col sep=comma]{\detokenize{4_step_L.csv}};
	\end{axis}
	\end{tikzpicture}
	\caption{Błąd replikacji prądu wyrażony w LSB dla różnych długości.}
\end{figure}

	
	\begin{wniosek}{ Błąd replikacji jest mniejszy dla tranzystorów o większej długości.}
	\end{wniosek}

	\subsection{Wstępny wybór L i W.}
	
		{ Wykonano symulację DC dla następujących parametrów:
		\begin{itemize}
			\item $A_d = 100M$, $V_{off} = 0V$
			\item $I_{ref} = 100nA$, $V_{ref} = 900mV$
			\item $L_0 = 10.8\mu$, $W_0 = 1\mu$
	\end{itemize} }
	
	
	\begin{figure}[!htb]
		\centering
		\begin{tikzpicture}
		\begin{axis}[
		width=0.6\linewidth, % Scale the plot to \linewidth
		grid=major, % Display a grid
		grid style={dashed,gray!30}, % Set the style
		xlabel=$ I_{ref} $, % Set the labels
		ylabel=$ \delta I$,
		legend style={at={(0.5,-0.2)},anchor=north}, % Put the legend below the plot
		x tick label style={anchor=north} % Display labels sideways
		]
		\addplot[only marks, mark=star, blue] table [x={I}, y={error}, col sep=comma]{\detokenize{5_final_LW.csv}};
		\end{axis}
		\end{tikzpicture}
		\caption{Błąd replikacji prądu wyrażony w LSB dla wybranego W/L.}
	\end{figure}
	
	
	\begin{wniosek}{ Dla całego zakresu prądu błąd jest poniżej połowy LSB.}
	\end{wniosek}
	
	\subsection{Sizing klucza.}

	{ Wykonano symulację DC dla następujących parametrów:
		\begin{itemize}
			\item $A_d = 100M$, $V_{off} = 0V$
			\item $I_{ref} = 100nA$, $V_{ref} = 900mV$
			\item $L_0 = 10.8\mu$, $W_0 = 1\mu$
			\item $W_K = 1\mu$, $L_K = 0.18\mu$
	\end{itemize} }
	
	
	\begin{figure}[!htb]
		\centering
		\begin{tikzpicture}
		\begin{axis}[
		width=0.6\linewidth, % Scale the plot to \linewidth
		grid=major, % Display a grid
		grid style={dashed,gray!30}, % Set the style
		xlabel=$ I_{ref} $, % Set the labels
		ylabel=$ \delta I$,
		legend style={at={(0.5,-0.2)},anchor=north}, % Put the legend below the plot
		x tick label style={anchor=north} % Display labels sideways
		]
		\addplot[only marks, mark=star, blue] table [x={error (WK=5e-07) X}, y={error (WK=5e-07) Y}, col sep=comma]{\detokenize{6_step_WK.csv}};
		\addplot[only marks, mark=star, red] table [x={error (WK=7.5e-07) X}, y={error (WK=7.5e-07) Y}, col sep=comma]{\detokenize{6_step_WK.csv}};
		\addplot[only marks, mark=star, black] table [x={error (WK=1e-06) X}, y={error (WK=1e-06) Y}, col sep=comma]{\detokenize{6_step_WK.csv}};
		\end{axis}
		\end{tikzpicture}
		\caption{Błąd replikacji prądu wyrażony w LSB dla wybranego W/L.}
	\end{figure}
	
	\subsection{Symulacja kaskady x2}
	{ Wykonano symulację DC dla następujących parametrów:
		\begin{itemize}
			\item $A_d = 100M$, $V_{off} = 0V$
			\item $I_{ref} = 100nA$, $V_{ref} = 900mV$
			\item $L_0 = 10.8\mu$, $W_0 = 1\mu$
			\item $W_K = 1\mu$, $L_K = 0.18\mu$
	\end{itemize} }
	
	
	\begin{figure}[!htb]
		\centering
		\begin{tikzpicture}
		\begin{axis}[
		width=0.6\linewidth, % Scale the plot to \linewidth
		grid=major, % Display a grid
		grid style={dashed,gray!30}, % Set the style
		xlabel=$ I_{ref} $, % Set the labels
		ylabel=$ \delta I$,
		legend style={at={(0.5,-0.2)},anchor=north}, % Put the legend below the plot
		x tick label style={anchor=north} % Display labels sideways
		]
		\addplot[only marks, mark=star, blue] table [x={I}, y={error}, col sep=comma]{\detokenize{7_sweep_iref_double.csv}};
		\end{axis}
		\end{tikzpicture}
		\caption{Błąd replikacji prądu wyrażony w LSB dla wybranego W/L. Czy to jest dobre .csv powtorzyć.}
	\end{figure}
	
	\begin{wniosek}{}
		\begin{equation}
			\delta I_{cascade}(I_{ref}) >> \delta I_{single}(I_{ref}) + \delta I_{single}(16*I_{ref}) 
		\end{equation}
	\end{wniosek}

	\subsection{Wpływ wartości wzmocnienia różnicowego idealnego wzmacniacza na błąd replikacji prądu.}
	\subsection{Wpływ wartości wejściowego napięcia niezrównoważenia idealnego wzmacniacza na błąd replikacji prądu.}
	\subsection{Powierzchnia konwertera.}
	{	Na tym etapie warto zauważyć, że powierzchnia tranzystorów w lustrze prądowym staje się znacząca. Zakładając optymistycznie, że rozmiar tranzystora to wyłącznie powierzchnia kanału $A=W_0\cdot L_0$, powierzchnia luster prądowych 4 bitowego przetwornika wynosi:
	\begin{equation}
		A_{4bit} =  \left( 1 + \sum_{i=0}^{3}2^i \right)   W_0 \cdot L_0= 16 W_0 \cdot L_0
	\end{equation}
	Dla przetwornika 8 bitowego:
	\begin{equation}
		A_{8bit} = \left( 1 + \sum_{i=0}^{7}2^i \right)   W_0 \cdot L_0 = 256 W_0 \cdot L_0
	\end{equation}
	Zastosowanie przetwornika segmentowego pozwala znacznie obniżyć powierzchnię do:
	\begin{equation}
		A_{8bit,seg} = 2\left( 1 + \sum_{i=0}^{3}2^i \right)   W_0 \cdot L_0 = 32 W_0 \cdot L_0
	\end{equation}
	Jeżeli ta powierzchnia okazuje się nieakceptowalna ze względu na cenę lub dostępną wolną powierzchnią w większym systemie, należy rozważyć inne architektury.
	}
	
	\section{Projekt wzmacniacza operacyjnego}
	
	\subsection{Wymagania.}
	{	Z badania przetwornika z idealnymi wzmacniaczami można wyciągnąć wnioski przydatne w sformułowaniu wymagań na rzeczywisty wzmacniacz. Napięcie $V_{DS}$ tranzystora M1 jest proporcjonalne do prądu referencyjnego $I_{ref}$, więc projektant dobierając prąd referencyjny musi zwrócić uwagę na to, że napięcie wyjściowe wzmacniacza będzie równe:
		\begin{equation}
			V_{out,A1} = V_{ref} - V_{DS}
		\end{equation} 
	\begin{przyklad}{W układzie lustra prądowego wybrano napięcie referencyjne jako $V_{ref} = 900mV$. Dla tranzystora o jednostkowych wymiarach, jaki jest maksymalny prąd referencyjny $I_{ref}$, dla którego napięcie wyjściowe wzmacniacza nie zejdzie poniżej $200mV$?}
		{Przekształcając powyższe równanie:
			\begin{equation}
				V_{DS} = V_{ref} - V_{out,A1} = 700mV
			\end{equation}
		Z równania na prąd drenu tranzystora w obszarze nasycenia uzyskujemy:
			\begin{equation}
				I_{DS} = K_n \frac{W}{L} \left(V_{DS} - V_{th,n}\right)^2 = 56 \mu \left(700m - 355m\right)^2 = 6.67 \mu A
			\end{equation}	
}
	\end{przyklad}
	\subsection{Wzmacniacz.}
	{	Zaprojektowany wzmacniacz operacyjny to prosty, dwustopniowy wzmacniacz, znany z literatury. Metodologia projektowania została szczegółowo opisana w \cite{cmosanal}, dlatego w tej pracy przytoczono tylko rezultaty. Zwraca się uwagę, że obciążenie pary różnicowej jest niesymetryczne, co powoduje, że wzmacniacz charakteryzuje się niezerowym systematycznym wejściowym napięciem niezrównoważenia. }
	{	Rozmiary tranzystorów zaprojektowanego wzmacniacza operacyjnego:
		\begin{itemize}
			\item $L = 1\mu$,
			\item $S1 = S2 = 10$, 
			\item $S3 = S4 = 10$,
			\item $S5 = 10$,
			\item $S6 = 10$,
			\item $S7 = 10$,
			\item $S8 = 10$,
			\item $S9 = 10$.
		\end{itemize}
		Wykonano pomiary podstawowych parametrów wzmacniacza:
		\begin{itemize}
			\item $ V_{off,systematic} = 300 \mu$,
			\item $ ICMR = $,
			\item $ V_{out,range} = $,
			\item $ A_{v,0} =  $,
			\item $ CMRR $,
			\item $ PSRR $,
			\item $ PM $,
			\item $ GB $,
			\item $ SR $


		\end{itemize}
	}






	\subsection{Analiza czasowa.}
	{ Na tym etapie warto wykonać analizę czasową, aby ocenić czas ustalania sygnału do pożądanej wartości. Wykonano analizę czasową dla pobudzenia kluczy skokiem.}
	{ Parametry badanego układu }
	{ Parametry sygnałów pobudzających}
	{ Odpowiedź}
	\begin{wniosek}
		Topologia z pojedynczym kluczem jest zdecydowanie za wolna, ponieważ trzeba przeładować pojemność od drenu tranzystorów w lustrze prądowym.
	\end{wniosek}
	{ Zaproponowano bardziej złożony klucz, którego główną wadą jest, że niezależnie od tego, czy klucz jest włączony, czy nie, pobierana jest taka sama moc przez układ.}
	
		\begin{figure}[!htb]
		\centering
		\begin{circuitikz}[scale = 0.6]
			%\draw [gray, color=gray] (0,-1) grid (20,10);
			\draw [color=black, thick]
			% Devices
			(4,1) node[nmos, label={ [centered,xshift=10, yshift = 10] {$M$} } ](m){}
			(3,4) node[nmos, label={ [centered,xshift=10, yshift = 10] {$K_n$} } ](kn){}
			(5,4) node[pmos, rotate = 180, label={ [centered,xshift=20, yshift = 10] {$K_p$} } ](kp){};

			\draw[color=black, thick]
			% Ports
			(m.D) to [short](4,2.5)
			(3,2.5) to [short](5,2.5)
			(kn.S) to [short](3,2.5)
			(kp.S) to [short](5,2.5)
			(kn.G) to [short, o-](2,4)
			(kp.G) to [short, -o](7,4)
			(kp.D) to [short, -o](5,6)
			(kn.D) to [short, -o](3,6)
			;
			
		\end{circuitikz}
		\caption{Para różnicowa jako klucz.}
		\end{figure}
	
	
	{ Rysunek nowej topologii.}
	\section{Podsumowanie projektu wstępnego.}
	{ Końcowy schemat zaprojektowanego przetwornika przedstawiono:}
	\begin{wniosek}{Zwiększanie prądu referencyjnego oznacza większy $V_{DS,sat}$, co z kolei oznacza, że wyjście wzmacniacza $A_1$ musi się obniżyć mocniej. To oznacza trudniejsze wymaganie na zakres napięć wyjściowych.}
	\end{wniosek}
	\begin{wniosek}{Zwiększanie napięcia referencyjnego, aby skompensować niskie wyjście wzmacniacza $A_1$ oznacza, że wyjście wzmacniacza $A_2$ będzie musiało z kolei pójść mocniej w górę.}
	\end{wniosek}
	\begin{wniosek}{Segmentowe połączenie pozwala nie tylko zmniejszyć minimalne L0, ale też łączną liczbę tranzystorów, potrzebnych do wykonania zadania.}
	\end{wniosek}
	\begin{wniosek}{}
	\end{wniosek}

	\chapter{Pomiary parametrów konwertera.}
	{ Do przeprowadzenia pomiarów konwertera przygotowano modele w Verilog-AMS, których szczegółowy opis znajduje sie w dodatku do pracy:}
	\begin{itemize}
		\item idealny 8 bitowy przetwornik cyfrowo analogowy,
		\item cyfrowy generator 8 bitowych słów,
		\item idealny 10-bitowy ADC,
		\item cyfrowy generator sygnału zegarowego.
	\end{itemize}
	\section{Pomiar nieliniowości całkowej INL.}
	{ Do pomiaru nieliniowości całkowej należy obliczyć błąd nieliniowości dla wszystkich $2^N$ statycznych stanów przetwornika. W tym celu przygotowano konfigurację jak na rysunku, ale nie zaznaczono generatora sygnału zegarowego, który generuje sygnał o częstotliwości $f_{clk} = 4MHz$. }
		\begin{figure}[!htb]
		\centering
		\begin{tikzpicture}[auto]
		\node [block, node distance = 2cm] (gen) {Generator cyfrowy};
		\node [block, below right of = gen, node distance = 4cm] (idealconv) {Idealny 8-bit DAC};
		\node [block, below left of = gen, node distance = 4cm] (dutconv) {Badany 8-bit DAC};
		\node [block, below of = gen, node distance = 6 cm](inl){Obliczenie błędu.};
		\node [signal, below of = inl, node distance = 2cm] (out) {INL/DNL};
		
		\draw [->] (gen) -- node {} (idealconv);
		\draw [->] (gen) -- node {} (dutconv);
		\draw [->] (idealconv) -- node {} (inl);
		\draw [->] (dutconv) -- node {} (inl);
		\draw [->] (inl) -- node {} (out);
		\end{tikzpicture}
		\caption{Schemat blokowy układu do pomiaru INL i DNL.}
		\end{figure}
	

	\section{Pomiar nieliniowości różniczkowej DNL.}
	Do pomiaru wykorzystano analogiczny układ jak do pomiaru INL.
	\section{Pomiar zniekształceń harmonicznych.}
	
	\begin{figure}[!htb]
		\centering
		\begin{tikzpicture}[auto]
		\node [block,  node distance = 2cm] (gen) {Generator sygnału sinusoidalnego};
		\node [block,  below of = gen, node distance = 2cm] (adc) {ADC};
		\node [block,  below of = adc, node distance = 2cm] (dutconv) {Badany 8-bit DAC};
		\node [block,  below of = dutconv, node distance = 2cm] (sh) {Układ próbkująco-pamiętający};
		\node [block,  below of = sh, node distance = 2cm] (dp) {Filtr DP};
		\node [block,  below of = dp, node distance = 2cm] (err) {Obliczenie błędu};
		\node [signal, below of = err, node distance = 2cm] (out) {THD};

		\draw [->] (gen) -- node {} (adc);
		\draw [->] (adc) -- node {} (dutconv);
		\draw [->] (dutconv) -- node {} (sh);
		\draw [->] (sh) -- node {} (dp);
		\draw [->] (dp) -- node {} (err);
		\draw [->] (err) -- node {} (out);
		\end{tikzpicture}
		\caption{Schemat blokowy układu do pomiaru THD.}
	\end{figure}
	\section{Charakterystyka przetwarzania.}
	\chapter{Projekt fizyczny.}
	\section{Widok topografii układu luster prądowych.}
	\section{Widok topografii wzmacniacza operacyjnego.}
	\section{Widok topografii całości konwertera.}
	
	\chapter{Pomiary parametrów konwertera po ekstrakcji.}
	\section{Pomiar nieliniowości całkowej INL.}
	\section{Pomiar nieliniowości różniczkowej DNL.}
	\section{Pomiar zniekształceń harmonicznych.}
	\section{Charakterystyka przetwarzania.}

	\chapter{Podsumowanie.}
	\section{Rozmiar układu, liczba tranzystorów.}
	\section{Weryfikacja osiągniętych celów.}
	\section{Wnioski końcowe.}	


\begin{thebibliography}{9}
	%%moje z knovela
	\bibitem{integconv} 
	Jespers, Paul G.A.. (2001). 
	\textit{Integrated Converters - D to A and A to D Architectures, Analysis and Simulation.}
	Oxford University Press
	
	\bibitem{cmosanal} 
	Allen, Phillip E. Holberg, Douglas R.. (2012)  
	\textit{CMOS Analog Circuit Design (3rd Edition). }
	Oxford University Press.
	
	\bibitem{vlsidesign} 
	Das, Debaprasad. (2015).
	\textit{VLSI Design (2nd edition)}
	Oxford University Press.
	%%z opisu tematu od profa allen cmos sie pokrywa z moim wyborem
	\bibitem{plassche} 
	R. Plassche (2001). 
	\textit{Scalone przetworniki analogowo-cyfrowe i cyfrowo-analogowe,}
	WKŁ 2001
	
	\bibitem{bdgp_1}
	Design of an improved Bandgap Reference in
	180nm CMOS Process Technology
	
	\bibitem{bdgp_2}
	Design and implementation of curvature corrected
	bandgap voltage reference - 1.1V using 180nm Technology
	
	\bibitem{ams_proc_params}
	$0.18\mu m$ HV CMOS Process Parameters, Rev. 3.0, AMS AG
	
	\bibitem{ams_match_params}
	$0.18\mu m$ HV CMOS Matching Parameters, Rev. 1.0, AMS AG
	
\end{thebibliography}
	\listoffigures

	\listoftables

	\appendix
	%insert chapters	
.	\chapter{Modele AMS}
	\section{Idealny 8-bit DAC.}
	\section{Idealny 10-bit ADC.}
	\section{Generator cyfrowy.}
	
	\chapter{Weryfikacja doboru długości kanału na podstawie charakterystyk niedopasowania.}
	{	Producent technologii dostarcza wzory opisujące odchylenie standardowe parametrów urządzeń, które zostały zaprojektowane jako identyczne. Do generacji prądów o binarnych wagach wykorzystujemy lustra prądowe, składające się z kolejno $1, 2, 4,..., 2^{N-1}$. Błąd napięciowego sygnału wyjściowego musi być mniejszy niż $\frac{LSB}{2}$. Zakładając, że napięciowy sygnał wejściowy jest proporcjonalny do sumy tych prądów:
		\begin{equation}
		V_{out} = A \sum_{i=0}^{N-1} b_{i}I_{i}
		\end{equation}
		LSB dla konwertera wynosi, gdzie FS oznacza pełen zakres sygnału wyjściowego:
		\begin{equation}
		LSB_v = \frac{FS_Vout}{2^{N-1}}
		\end{equation}
		Dla prądów:
		\begin{equation}
		LSB_i = \frac{LSB_v}{A}
		\end{equation}
		
		W związku z tym faktem zakłada się, że błąd każdego z prądów również musi być mniejszy od $\frac{LSB}{2}$. Prąd i-tej gałęzi jest sumą $2^{i}$ prądów, przepływających przez $2^{i}$ identycznych tranzystorów. Aby błąd sumarycznego prądu każdej gałęzi był mniejszy niż LSB/2, suma błędów niedopasowania wszystkich tranzystorów w każdej gałęzi musi być mniejsza niż LSB/2. 
		\begin{equation}
		\sigma \left( \Delta I_{i,ds} \right) < \frac{LSB_i}{2}
		\end{equation}
		Łatwo zauważyć, że największy błąd jest dla ostatniej gałęzi z największą liczbą tranzystorów: $n=2^{N-1}=2^7$. Minimalny prąd referencyjny w pierwszej gałęzi ma wartość LSB.
		\begin{equation}
		n \sigma \left( \Delta I_{N-1,ds} \right) LSB_i = 2^{N-1} \sigma \left( \Delta I_{N-1,ds} \right) LSB_i < \frac{LSB_i}{2}
		\end{equation}
		
		\begin{equation}
		\sigma \left( \Delta I_{N-1,ds} \right) < 2^{-N}
		\end{equation}
		
		Z \cite{ams_match_params} mamy:
		\begin{equation}
		\sigma^2 \left( \frac{\Delta I_{ds}}{I_{ds}}\right) = 	\sigma^2 \left( \frac{\Delta \mu}{\mu}\right) + \sigma^2 \left( \frac{2\Delta V_T}{V_{gs} - V_{T}}\right)
		\end{equation}
		
		\begin{equation}
		3 \sigma \left(\Delta \mu \right) = \frac{ \sqrt{2} A_{\beta}  } { \sqrt{ \left(W-A_{\beta W}\right)\left(L-A_{\beta} L\right)}}
		\end{equation}
		
		\begin{equation}
		3 \sigma \left(\Delta V_T \right) = \frac{ \sqrt{2} A_{V_T}  } { \sqrt{ \left(W-A_{VTW}\right)\left(L-A_{VTL}\right)}}
		\end{equation}
		Wykreślmy błąd prądu przy pomocy Octave, znajdziemy minimalne L.
		
	}
	\chapter{Instrukcja uruchomienia symulacji AMS w środowsiku Cadence2012/13.}
	{	W tym dodatku przedstawiono serię kroków, które należy wykonać, aby uruchomić symulację dla sygnałów mieszanych (Analog Mixed Signal) w środowsisku Cadence2012/13. Tego typu symulacja pozwala jednoczesną symulację układów elektronicznych z dziedziny analogowej i cyfrowej. Układy analogowe mogą być opisane przy pomocy języków: SPECTRE, SPICE, Verilog-A, natomiast układy cyfrowe mogą być opisane przy pomocy klasycznego Verilog lub VHDL. Układy o mieszanej naturze opisywane są w języku Verilog-AMS.}
	{	Symulacja układu mieszanego wymaga dostępu do biblioteki elementów połączeniowych, konwertujących sygnał z postaci logicznej do elektrycznej (L2E) i z postaci elektrycznej do logicznej (E2L). Zazwyczaj producenci technologii dostarczają taką bibliotekę.}
	\section{Organizacja plików wejściowych.}
	{	W opisanej tu procedurze zakłada się, że najwyższą w hierarchii komórką jest schemat elektryczny, przygotowany przy pomocy "Schematic Editor". Pozostałe komórki są instancjowane przy pomocy symboli. Dopuszcza się też umieszczanie prymitywów (idealne źródła napięciowe, etc.) na schemacie.}
	\subsection{Przygotowanie komórki Verilog-AMS.}
	\begin{enumerate}
		\item Utworzenie komórki Verilog-AMS \\
			W \textbf{Library Manager} z górnej belki należy wybrać: \\ 
			\textbf{File} $\rightarrow$ \textbf{Create} \\
			W polu \textbf{Cell name} wpisać nazwę komórki.
		\item Wpisac kod.
		\item Skompilować.
	\end{enumerate}
	\subsection{Przygotowanie schematu.}
	
	\subsection{Przygotowanie widoku konfiguracji.}
	\subsection{Przygotowanie komórki do testowania ADE.}
	
	
	\subsection{Konfiguracja symulatora AMS.}
	
	\subsection{Symulacja w SimVision.}
	
	
	\chapter{Inne rzeczy, ktore nie wiadomo gdzie maja trafić}
		\subsection{Konwersja prądu na napięcie przy pomocy pojedynczego tranzystora.}
	{	Tranzystor MOSFET połączony jak na \ref{itov} może być traktowany jako przetwornik prądu na napięcie. 
	}
	
	\begin{figure}[h!]
		\begin{center}
			\begin{circuitikz}
				\draw [color=black, thick]
				%Ground
				(0,0) node[ground]{} 
				(0,0) to [I=${I_{ref}}$] (0,2)
				(2,0) node[ground]{} 
				(2,1) node[nmos, rotate =180, label={ [centered,xshift=-10, yshift = 10] {$M_0$} } ](m0){}
				(0,2) to[short] (2,2)
				(2,0) to[short] (m0.D)
				(2,2) to[] (m0.S)
				(2,2) to[short,*-] (3,2)
				(3,2) to[short,-*] (3,1)
				%Ground line
				(m0.G) to[short,-o] (4,1)
				(4.5,1) node[]{\large{\textbf{$V_{out}$}}}
				;
			\end{circuitikz}
			\caption{Konwersja prądu na napięcie.}
			\label{itov}
		\end{center}	
	\end{figure}
	
	
	{	Zakładając, że tranzystor pracuje w obszarze aktywnym, prąd drenu jest równy:
		\begin{equation}
		I_{REF} = K' \frac{W}{L}\left( V_{GS}-V_{TH,n}\right)^2
		\end{equation}
		Zaniedbując efekt modulacji kanału i uwzględniając, że źródło tranzystora jest na poziomie masy sygnałowej, napięcie wyjściowy może zostać wyliczone jako:
		\begin{equation} \label{eq_v_out_mosfet}
		V_{OUT} = V_{D} = V_{G} = \sqrt{\frac{I_{REF}}{K'}\frac{L}{W}} + V_{TH,n}
		\end{equation}
	}
	
	\subsection{Konwersja prądu na napięcie przy pomocy tranzystora i wzmacniacza operacyjnego.}
	
	\begin{figure}
		\centering
		\begin{circuitikz}
			\draw [color=black, thick]
			%Ground
			
			(0, 1) node[op amp, rotate =270] (opamp) {$A_1$}
			(opamp.+) node[ground, rotate = 180] {}
			
			(opamp.-) to [short] (0.5,2.5)
			(2,5) to [I=${I_{ref}}$] (2,3)
			(2,5) node[ground, rotate = 180 ]{}
			(2,1) node[nmos, rotate =180, label={ [centered,xshift=-10, yshift = 10] {$M_0$} } ](m0){}
			(0.5,2.5) to[short] (2,2.5)
			(2,0) to[short] (m0.D)
			(opamp.out) to [short](0,-0.5)
			(2,3) to [] (m0.S)
			(0,-0.5) to [short] (2,-0.5)
			(m0.D) to [] (2,-0.5)
			(2,2) to[] (m0.S)
			(2,2.5) to[short,*-] (3,2.5)
			(3,2.5) to[short,-*] (3,1)
			%Ground line
			(m0.G) to[short,-o] (4,1)
			(4.5,1) node[]{\large{\textbf{$V_{out}$}}}
			;
		\end{circuitikz}
		\caption{Konwersja z wzmacniaczem.}
	\end{figure}
	
	
	
	
	{ Wzmacniacz operacyjny w pętli sprzężenia zwrotnego wymusi potencjał masy na wejściu minus. Biorąc dane z modelu do obliczeń ręcznych i dla rysunku powyżej, z \ref{eq_cutoff} otrzymujemy nierówność:
		$$
		0 - V_s > V_{th,n} = 0.4
		$$
		Źródło prądowe wymusza prąd $I_{ref}$, więc:
		$$
		I_{ref} = K \frac{W}{L}\left( V_s-V_{th,n}\right)^2
		$$
		Zakładając jednostkowy stosunek szerkości do wyskości i podstawiając dane otrzymujemy dla $I_{ref} = 10\mu A$:
		$$
		V_s = -0.839V
		$$
	}

	\subsection{Źródło napięcia referencyjnego zbudowane z dwóch tranzystorów.}
obwód generacji napięcia odniesienia jak z rysunku...
Ze względu na połączenie drenu i bramki, tranzystor może pracować tylko w nasyceniu lub odcięciu. Zakładając, że oba tranzystory
są w nasyceniu i oznaczając napięcie drenu jako $V_{out}$ układamy równania dla nMOSFET:
\begin{equation}
V_{out} > V_{th,n}
\end{equation}

\begin{equation}
V_{out} > V_{out} - V_{th,n}
\end{equation}

i pMOSFET:

\begin{equation}
V_{dd} - V_{out} > \left|V_{th,p}\right|
\end{equation}

\begin{equation}
V_{dd} - V_{out} > V_{dd} - V_{out} - \left|V_{th,p}\right|
\end{equation}

Po prostych przekształceniach:
\begin{equation}
V_{out} > V_{th,n}
\end{equation}

\begin{equation}
0 > - V_{th,n}
\end{equation}

i pMOSFET:

\begin{equation}
V_{dd} - V_{out} > \left|V_{th,p}\right|
\end{equation}

\begin{equation}
0 > - \left|V_{th,p}\right|
\end{equation}


Ponadto, prądy muszą być równe:
\begin{equation}
I_{d,n} = I_{d,p}
\end{equation}

W nasyceniu:
\begin{equation}
\beta_n \left(V_{out} - V_{th,n}\right)^2 = \beta_p \left(V_{dd} - V_{out} - \left|V_{th,p}\right|\right)^2
\end{equation}
Obie strony są zawsze dodatnie, więc możemy pierwiastkować:
\begin{equation}
\left(V_{out} - V_{th,n}\right) = \sqrt{\frac{\beta_p}{\beta_n}}\left(V_{dd} - V_{out} - \left|V_{th,p}\right|\right)
\end{equation}

\begin{equation}
V_{out}\left(1+\sqrt{\frac{\beta_p}{\beta_n}}\right) = \sqrt{\frac{\beta_p}{\beta_n}}\left(V_{dd} -\left|V_{th,p}\right|\right)+V_{th,n}
\end{equation}

\begin{equation}
V_{out} =\frac{\sqrt{\frac{\beta_p}{\beta_n}}\left(V_{dd} -\left|V_{th,p}\right|\right)+V_{th,n}}{\left(1+\sqrt{\frac{\beta_p}{\beta_n}}\right)}
\end{equation}	

Zauważmy, że dla procesu o równych napięciach progowych i tranzystorów o tej samej $\beta$, wynik jest dość intuicyjny:
\begin{equation}
V_{out} =\frac{1}{2} V_{dd}
\end{equation}

\begin{circuitikz}
	\draw [color=black, thick]
	(0, 0) node[op amp] (opamp) {}
	(opamp.-) to[R] (-3, 0.5)
	(opamp.-) to[short,*-] ++(0,1.5) coordinate (leftR)
	to[R] (leftR -| opamp.out)
	to[short,-*] (opamp.out)
	;
\end{circuitikz}


\end{document}
